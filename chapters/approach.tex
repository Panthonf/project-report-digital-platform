\chapter{\ifproject%
      \ifenglish Project Structure and Methodology\else โครงสร้างและขั้นตอนการทำงาน\fi
  \else%
      \ifenglish Project Structure\else โครงสร้างของโครงงาน\fi
  \fi
 }

ในบทนี้จะกล่าวถึงหลักการ และการออกแบบระบบ

\makeatletter

% \renewcommand\section{\@startsection {section}{1}{\z@}%
%                                    {13.5ex \@plus -1ex \@minus -.2ex}%
%                                    {2.3ex \@plus.2ex}%
%                                    {\normalfont\large\bfseries}}

\makeatother
%\vspace{2ex}
% \titleformat{\section}{\normalfont\bfseries}{\thesection}{1em}{}
% \titlespacing*{\section}{0pt}{10ex}{0pt}

\section{หลักการทำงานของระบบ}


% \begin{figure}
% \begin{center}
% \includegraphics{800px-Briny_Beach.jpg}
% \end{center}
% \caption[Poem]{The Walrus and the Carpenter}
% \label{fig:walrus}
% \end{figure}
\subsection{การทำงานของระบบ (System Architecture)}
\begin{figure}[h]
    \begin{center}
        \includegraphics[width=0.9\textwidth]{img/system achitecture.png}
    \end{center}
    \caption{System Architecture}
    \label{fig:system_overview}
\end{figure}

จากรูปที่ \ref{fig:system_overview} จะเป็นภาพรวมของระบบที่เราได้ทำการออกแบบขึ้น โดยมีรายละเอียดดังนี้
\begin{itemize}
    \item Frontend

          ส่วนหน้าบ้าน (Frontend) เป็นส่วนการพัฒนาเพื่อแสดง User Interface (UI) โดยโครงการนี้ได้ใช้เทคโนโลยี React ร่วมกับ Mantine UI ในการออกแบบและสร้างส่วนติดต่อผู้ใช้ (UI) โดยมีหน้าที่แสดงผลลัพธ์ต่อผู้ใช้ในรูปแบบที่เข้าใจง่าย รับข้อมูลป้อนจากผู้ใช้ผ่านอินเทอร์เฟซต่าง ๆ เช่น ปุ่ม ฟิลด์ข้อความ ฯลฯ และสื่อสารกับ API เพื่อส่งคำร้องขอและรับผลลัพธ์
    \item Backend

          ส่วนหลังบ้าน (Backend) เป็นส่วนที่ทำหน้าที่รับข้อมูลจากผู้ใช้ จากนั้นทำการประมวลผลข้อมูล และส่งข้อมูลกลับไปยังผู้ใช้ โดยโครงการนี้ได้ใช้เทคโนโลยี Fastify เว็บเฟรมเวิร์ค Node.js ร่วมกับ Prisma ORM ในการจัดการข้อมูลในฐานข้อมูล รวมถึงทำการสื่อสารกับฐานข้อมูล PostgreSQL และ Minio (Object Storage) ในการจัดการข้อมูลที่เป็นไฟล์
\end{itemize}

\subsection{เส้นทางของผู้ใช้ (User Flow)}
\begin{figure}[h]
    \begin{center}
        \includegraphics[width=0.9\textwidth]{img/userflow.png}
    \end{center}
    \caption{User Flow}
    \label{fig:user_flow}
\end{figure}

จากรูปที่ \ref{fig:user_flow} จะเป็นเส้นทางของผู้ใช้งานที่เข้าใช้งานระบบ โดยมีรายละเอียดดังนี้

\subsubsection{ผู้สร้างงานกิจกรรม (Event Manager)}
\begin{itemize}
    \item ผู้ใช้เข้าสู่ระบบโดยใช้ Google Account
    \item ผู้ใช้สร้าง Event ใหม่ โดยป้อนข้อมูลต่างๆ เช่น ชื่องานกิจกรรม รายละเอียด วันเวลา สถานที่ เป็นต้น
    \item ผู้ใช้สร้าง Event เสร็จสิ้น จะมี QR Code หรือ Access Link สำหรับนำไปให้ผู้นำเสนอโครงการ (Presenter) สำหรับเพิ่มโครงการของตนและผู้เข้าร่วมกิจกรรม (Guest) สำหรับเข้าร่วมกิจกรรม
    \item ผู้ใช้สามารถดูผลลัพธ์ (Chart และ Ranking) ของโครงการต่าง ๆ ที่เข้าร่วมกิจกรรมในระหว่างจัดงานกิจกรรม

\end{itemize}

\subsubsection{ผู้นำเสนอโครงการ (Presenter)}
\begin{itemize}
    \item ผู้ใช้เข้าสู่ระบบโดยใช้ Google Account
    \item ผู้ใช้สร้าง Project ใหม่ โดยป้อนข้อมูลต่าง ๆ เช่น ชื่อโครงการ รายละเอียด รูปภาพ ลิงก์ และไฟล์อื่น ๆ ที่เกี่ยวข้อง
    \item ผู้ใช้ดูผลลัพธ์ว่าโครงการของตนได้รับ Virtual Money จากผู้เข้าร่วมกิจกรรม (Guest) และความคิดเห็นเกี่ยวกับโครงการของตนหลังจากเสร็จสิ้นงานกิจกรรม

\end{itemize}

\subsubsection{ผู้เข้าร่วมกิจกรรม (Guest)}
\begin{itemize}
    \item ผู้ใช้เข้าร่วม Event โดยใช้ QR COde หรือ Access Link ที่ได้รับจาก Event Manager
    \item ผู้ใช้เข้าสู่ระบบโดยใช้ Google Account
    \item ผู้ใช้ดูรายละเอียดของงานกิจกรรมที่จัดขึ้น รวมถึงโครงการที่เข้าร่วมกิจกรรม
    \item ผู้ใช้สามารถให้ Virtual Money และแสดงความคิดเห็นเกี่ยวกับ Project ต่าง ๆ ที่เข้าร่วมกิจกรรมได้
\end{itemize}

\newpage
\subsection{โครงสร้างฐานข้อมูล (Database Schema)}
\begin{figure}[hb]
    \begin{center}
        \includegraphics[width=0.8\textwidth]{img/database schama.png}
    \end{center}
    \caption{Database Schema}
    \label{fig:data_schema}
\end{figure}
จากรูปที่ \ref{fig:data_schema} จะเป็นโครงสร้างของฐานข้อมูลที่ใช้ในระบบ โดยมีทั้งหมด 9  ตาราง ดังนี้


\begin{table}[hb]
    \centering
    \begin{tabular}{|c|c|c|}
        \hline
        ชื่อคอลัมน์               & ชนิดข้อมูล   & คำอธิบาย          \\ \hline
        \verb |first_name_th| & VARCHAR   & ชื่อภาษาไทย       \\ \hline
        \verb |last_name_th|  & VARCHAR   & นามสกุลภาษาไทย   \\ \hline
        \verb |first_name_en| & VARCHAR   & ชื่อภาษาอังกฤษ     \\ \hline
        \verb |last_name_en|  & VARCHAR   & นามสกุลภาษาอังกฤษ \\ \hline
        \verb |email|         & VARCHAR   & อีเมล์            \\ \hline
        \verb |affiliation|   & TEXT      & สังกัด            \\ \hline
        \verb |profile_pic|   & TEXT      & รูปภาพโปรไฟล์     \\ \hline
        \verb |role|          & VARCHAR   & บทบาทของผู้ใช้งาน  \\ \hline
        \verb |created_at|    & TIMESTAMP & วันที่สร้างข้อมูล     \\ \hline
        \verb |updated_at|    & TIMESTAMP & วันที่อัปเดตข้อมูล    \\ \hline
    \end{tabular}
    \caption{ตารางข้อมูลผู้ใช้งาน (Users)}
    \label{tab:user_data}
\end{table}

\begin{table}[ht]
    \centering
    \begin{tabular}{|c|c|c|}
        \hline
        ชื่อคอลัมน์                  & ชนิดข้อมูล   & คำอธิบาย                                 \\ \hline
        \verb |event_name|       & VARCHAR   & ชื่องานกิจกรรม                            \\ \hline
        \verb |start_date|       & DATE      & วันที่เริ่มงานกิจกรรม                        \\ \hline
        \verb |end_date|         & DATE      & วันที่สิ้นสุดงานกิจกรรม                       \\ \hline
        \verb |description|      & TEXT      & รายละเอียดงานกิจกรรม                     \\ \hline
        \verb |submit_start|     & DATE      & วันที่เริ่มรับส่งโครงการ                      \\ \hline
        \verb |submit_end|       & DATE      & วันที่สิ้นสุดรับส่งโครงการ                     \\ \hline
        \verb |number_of_member| & INTEGER   & จำนวนสมาชิกของโครงการ                    \\ \hline
        \verb |virtual_money|    & INTEGER   & จำนวนเงินเสมือนที่จะให้แขกผู้เข้าร่วมงาน (Guest) \\ \hline
        \verb |unit_money|       & VARCHAR   & หน่วยเงินเสมือนที่จะให้แขกผู้เข้าร่วมงาน (Guest) \\ \hline
        \verb |published|        & BOOLEAN   & สถานะการเผยแพร่ของงานกิจกรรม             \\ \hline
        \verb |organization|     & TEXT      & สังกัดของงานกิจกรรม                       \\ \hline
        \verb |video_link|       & TEXT      & ลิงก์วิดีโอที่เกี่ยวข้องกับงานกิจกรรม             \\ \hline
        \verb |user_id|          & INTEGER   & รหัสผู้ใช้งานที่สร้างงานกิจกรรม                \\ \hline
        \verb |location|         & TEXT      & สถานที่จัดงานกิจกรรม                       \\ \hline
        \verb |created_at|       & TIMESTAMP & วันที่สร้างข้อมูล                            \\ \hline
        \verb |updated_at|       & TIMESTAMP & วันที่อัปเดตข้อมูล                           \\ \hline
    \end{tabular}
    \caption{ตารางข้อมูลงานกิจกรรม (Events)}
    \label{tab:event_data}
\end{table}

\begin{table}[hb]
    \centering
    \begin{tabular}{|c|c|c|}
        \hline
        ชื่อคอลัมน์             & ชนิดข้อมูล   & คำอธิบาย                     \\ \hline
        \verb |title|       & VARCHAR   & ชื่อโครงการ                  \\ \hline
        \verb |description| & TEXT      & รายละเอียดของโครงการ        \\ \hline
        \verb |user_id|     & INTEGER   & รหัสผู้ใช้งานที่สร้างโครงการ      \\ \hline
        \verb |event_id|    & INTEGER   & รหัสงานกิจกรรมที่โครงการเข้าร่วม \\ \hline
        \verb |created_at|  & TIMESTAMP & วันที่สร้างข้อมูล                \\ \hline
    \end{tabular}
    \caption{ตารางข้อมูลโครงการ (Projects)}
    \label{tab:project_data}
\end{table}

\begin{table}[ht]
    \centering
    \begin{tabular}{|c|c|c|}
        \hline
        ชื่อคอลัมน์            & ชนิดข้อมูล   & คำอธิบาย                   \\ \hline
        \verb |amount|     & INTEGER   & จำนวนเงินเสมือนที่ได้รับ        \\ \hline
        \verb |project_id| & INTEGER   & รหัสโครงการที่ได้รับเงิน       \\ \hline
        \verb |event_id|   & INTEGER   & รหัสงานกิจกรรมที่ได้รับเงิน     \\ \hline
        \verb |guest_id|   & INTEGER   & รหัสผู้เข้าร่วมกิจกรรมที่ได้รับเงิน \\ \hline
        \verb |created_at| & TIMESTAMP & วันที่สร้างข้อมูล              \\ \hline
        \verb |updated_at| & TIMESTAMP & วันที่อัปเดตข้อมูล             \\ \hline
    \end{tabular}
    \caption{ตารางข้อมูลเงินเสมือน (Virtual Money)}
    \label{tab:virtual_money_data}
\end{table}

\begin{table}[hb]
    \centering
    \begin{tabular}{|c|c|c|}
        \hline
        ชื่อคอลัมน์               & ชนิดข้อมูล   & คำอธิบาย           \\ \hline
        \verb |first_name_th| & VARCHAR   & ชื่อจริงภาษาไทย     \\ \hline
        \verb |last_name_th|  & VARCHAR   & นามสกุลภาษาไทย    \\ \hline
        \verb |first_name_en| & VARCHAR   & ชื่อจริงภาษาอังกฤษ   \\ \hline
        \verb |last_name_en|  & VARCHAR   & นามสกุลภาษาอังกฤษ  \\ \hline
        \verb |email|         & TEXT      & ที่อยู่อีเมล          \\ \hline
        \verb |profile_pic|   & TEXT      & รูปภาพโปรไฟล์      \\ \hline
        \verb |virtual_money| & INTEGER   & จำนวนเงินเสมือนที่มีอยู่ \\ \hline
        \verb |created_at|    & TIMESTAMP & วันที่สร้างข้อมูล      \\ \hline
    \end{tabular}
    \caption{ตารางข้อมูลแขกผู้เข้าร่วมกิจกรรม (Guests)}
    \label{tab:guest_data}
\end{table}

\begin{table}[ht]
    \centering
    \begin{tabular}{|c|c|c|}
        \hline
        ชื่อคอลัมน์               & ชนิดข้อมูล   & คำอธิบาย                \\ \hline
        \verb |event_id|      & INTEGER   & รหัสงานกิจกรรมที่ได้รับเงิน  \\ \hline
        \verb |thumbnail|     & TEXT      & รูปภาพตัวอย่างโครงการ    \\ \hline
        \verb |thumbnail_url| & TEXT      & ลิงก์รูปภาพตัวอย่างโครงการ \\ \hline
        \verb |created_at|    & TIMESTAMP & วันที่สร้างข้อมูล           \\ \hline
        \verb |updated_at|    & TIMESTAMP & วันที่อัปเดตข้อมูล          \\ \hline
    \end{tabular}
    \caption{ตารางข้อมูลภาพตัวอย่างโครงการ (Thumbnails)}
    \label{tab:thumbnail_data}
\end{table}

\begin{table}[hb]
    \centering
    \begin{tabular}{|c|c|c|}
        \hline
        ชื่อคอลัมน์               & ชนิดข้อมูล   & คำอธิบาย                     \\ \hline
        \verb |project_id|    & INTEGER   & รหัสโครงการที่เกี่ยวข้องกับเอกสาร \\ \hline
        \verb |document_name| & TEXT      & ชื่อเอกสาร                   \\ \hline
        \verb |document_url|  & TEXT      & ลิงก์เอกสาร                  \\ \hline
        \verb |created_at|    & TIMESTAMP & วันที่สร้างข้อมูล                \\ \hline
        \verb |updated_at|    & TIMESTAMP & วันที่อัปเดตข้อมูล               \\ \hline
    \end{tabular}
    \caption{ตารางข้อมูลเอกสารที่เกี่ยวข้องกับงานกิจกรรม (Documents)}
    \label{tab:document_data}
\end{table}

\begin{table}[ht]
    \centering
    \begin{tabular}{|c|c|c|}
        \hline
        ชื่อคอลัมน์                & ชนิดข้อมูล   & คำอธิบาย                        \\ \hline
        \verb |project_id|     & INTEGER   & รหัสโครงการที่เกี่ยวข้องกับความคิดเห็น \\ \hline
        \verb |comment_better| & TEXT      & ความคิดเห็นที่ดีของโครงการ         \\ \hline
        \verb |comment_idea|   & TEXT      & ความคิดเห็นที่เสนอไอเดียใหม่        \\ \hline
        \verb |comment_ilike|  & TEXT      & ความคิดเห็นที่ชอบของโครงการ       \\ \hline
        \verb |created_at|     & TIMESTAMP & วันที่สร้างข้อมูล                   \\ \hline
        \verb |updated_at|     & TIMESTAMP & วันที่อัปเดตข้อมูล                  \\ \hline
    \end{tabular}
    \caption{ตารางข้อมูลความคิดเห็นของผู้เข้าร่วมกิจกรรม (Comments)}
    \label{tab:comment_data}
\end{table}

\begin{table}[hb]
    \centering
    \begin{tabular}{|c|c|c|}
        \hline
        ชื่อคอลัมน์                   & ชนิดข้อมูล   & คำอธิบาย                    \\ \hline
        \verb |project_id|        & INTEGER   & รหัสโครงการที่เกี่ยวข้องกับรูปภาพ \\ \hline
        \verb |project_image|     & TEXT      & รูปภาพโครงการ              \\ \hline
        \verb |project_image_url| & TEXT      & ลิงก์รูปภาพโครงการ           \\ \hline
        \verb |created_at|        & TIMESTAMP & วันที่สร้างข้อมูล               \\ \hline
        \verb |updated_at|        & TIMESTAMP & วันที่อัปเดตข้อมูล              \\ \hline
    \end{tabular}
    \caption{ตารางข้อมูลรูปภาพที่เกี่ยวข้องกับโครงการ (Project Images)}
    \label{tab:project_images_data}
\end{table}

\clearpage % 
\section{ส่วนเชื่อมต่อระหว่างผู้ใช้งานกับระบบ (User Interface)}


โดยแบ่งเป็น 3 ผู้ใช้งาน คือ ผู้สร้างงานกิจกรรม (Event manager), ผู้นำเสนอผลงาน (Presenter), ผู้เข้าร่วมงาน (Guest)
\subsection{หน้าเริ่มต้นการใช้งาน (Homepage)}
โดยแบ่ง Tab เป็น Home, Service, About us และ Contact สำหรับ Event manager และ Presenter

\begin{figure}[h!] %
    \begin{center}
        \includegraphics[width=0.9\textwidth]{img/ui/homepage.png}
    \end{center}
    \caption{หน้า Homepage}
    \label{fig:homepage}

    \begin{center}
        \includegraphics[width=0.9\textwidth]{img/ui/service.png}
    \end{center}
    \caption{หน้า Service}
    \label{fig:service}
\end{figure}


\begin{figure}
    \begin{center}
        \includegraphics[width=0.9\textwidth]{img/ui/aboutus.png}
    \end{center}
    \caption{หน้า About us}
    \label{fig:aboutus}

    \begin{center}
        \includegraphics[width=0.9\textwidth]{img/ui/contact.png}
    \end{center}
    \caption{หน้า Contact}
    \label{fig:contact}
\end{figure}

\subsection{หน้าเข้าสู่ระบบ (Login)}
Login ด้วย Google account สำหรับ Event manager และ Presenter

\begin{figure}[h!] %
    \begin{center}
        \includegraphics[width=0.9\textwidth]{img/ui/login.png}
    \end{center}
    \caption{หน้า Login สำหรับ Event manager และ Presenter}
    \label{fig:login}
\end{figure}

\subsection{ผู้สร้างงานกิจกรรม (Event manager)}
\begin{figure}[h!] %
    \begin{center}
        \includegraphics[width=0.9\textwidth]{img/ui/event-dashboard.png}
    \end{center}
    \caption{หน้าแสดงกิจกรรมที่สร้าง (Event manager dashboard)}
    \label{fig:event-dashboard}
\end{figure}

\begin{figure}
    \begin{center}
        \includegraphics[width=0.9\textwidth]{img/ui/event-info.png}
    \end{center}
    \caption{หน้าแสดงข้อมูลรายละเอียดของกิจกรรมที่สร้าง โดยแบ่ง Tab เป็น Event information, Project และ Result}
    \label{fig:event-info}

    \begin{center}
        \includegraphics[width=0.9\textwidth]{img/ui/project-list.png}
    \end{center}
    \caption{หน้าแสดงรายการโครงการที่เข้าร่วมกิจกรรม}
    \label{fig:project-list}
\end{figure}

\begin{figure}[t]
    \begin{center}
        \includegraphics[width=0.9\textwidth]{img/ui/result.png}
    \end{center}
    \caption{หน้าแสดงผลลัพธ์ของโครงการที่เข้าร่วมกิจกรรม โดยแสดงเป็น Ranking และตาราง Virtual Money}
    \label{fig:result}

    \begin{center}
        \includegraphics[height=0.7\textwidth]{img/ui/ranking-table-result.png}
    \end{center}
    \caption{ตารางแสดงผลลัพธ์ของโครงการที่เข้าร่วมกิจกรรม}
    \label{fig:ranking-table-result}
\end{figure}

\clearpage
\subsubsection{สร้างกิจกรรม (Create Event)}
\begin{figure}[h]
    \begin{center}
        \includegraphics[width=0.9\textwidth]{img/ui/create-event-1.png}
    \end{center}
    \caption{หน้าสร้างกิจกรรม First step (Event information)}
    \label{fig:create-event}
    \begin{center}
        \includegraphics[width=0.9\textwidth]{img/ui/create-event-2.png}
    \end{center}
    \caption{หน้าสร้างกิจกรรม Second step (Images Dropzone)}
    \label{fig:create-event-2}
\end{figure}

\clearpage
\begin{figure}[ht]
    \begin{center}
        \includegraphics[width=0.9\textwidth]{img/ui/create-event-3.png}
    \end{center}
    \caption{หน้าสร้างกิจกรรม Final step (Check all information before submit)}
    \label{fig:create-event-3}
\end{figure}

\clearpage
\subsection{ผู้นำเสนอผลงาน (Presenter)}
หลังจากที่เข้าสู่งานกิจกรรมจาก QR Code หรือ Access Link ที่ได้รับจาก Event manager จะเข้าสู่หน้าเข้าสู่ระบบ (Login) และหลังจากที่เข้าสู่ระบบจะเข้าสู่หน้าแสดงข้อมูลรายละเอียดของงานกิจกรรมที่จัดขึ้น และผู้นำเสนอโครการสามารถเพิ่มโครงการเข้าร่วมงานกิจกรรมได้
\begin{figure}[h]
    \begin{center}
        \includegraphics[width=0.9\textwidth]{img/ui/add-project.png}
    \end{center}
    \caption{หน้าเพิ่มโครงการที่เข้าร่วมกิจกรรม (Add project)}
    \label{fig:add-project}

    \begin{center}
        \includegraphics[width=0.9\textwidth]{img/ui/presenter-dashboard.png}
    \end{center}
    \caption{หน้าแสดงโครงการที่สร้าง (Presenter dashboard)}
    \label{fig:presenter-dashboard}
\end{figure}

\clearpage
\begin{figure}[ht]
    \begin{center}
        \includegraphics[width=0.9\textwidth]{img/ui/project-info.png}
    \end{center}
    \caption{หน้าแสดงข้อมูลรายละเอียดของผลงานที่สร้างในกิจกรรม}
    \label{fig:project-info}
\end{figure}
\clearpage %

% \clearpage
\subsection{ผู้เข้าร่วมงาน (Guest)}
เมื่อเข้าสู่งานกิจกรรมจาก QR Code หรือ Access Link ที่ได้รับจาก Event manager จะเข้าสู่หน้าเข้าสู่ระบบ (Login) และหลังจากที่เข้าสู่ระบบจะเข้าสู่หน้าแสดงข้อมูลรายละเอียดของงานกิจกรรมที่จัดขึ้น และโครงการที่เข้าร่วมกิจกรรม
\begin{figure}[h]
    \begin{center}
        \includegraphics[width=0.9\textwidth]{img/ui/guest-login.png}
    \end{center}
    \caption{หน้าเข้าสู่ระบบ (Login) สำหรับผู้เข้าร่วมกิจกรรม (Guest)}
    \label{fig:guest-dashboard}

    \begin{center}
        \includegraphics[height=0.5\textwidth]{img/ui/guest-dashboard.png}
    \end{center}
    \caption{หน้าแสดงข้อมูลรายละเอียดของงานกิจกรรมที่จัดขึ้น และโครงการที่เข้าร่วมกิจกรรม}
    \label{fig:guest-dashboard}
\end{figure}

\begin{figure}
    \begin{center}
        \includegraphics[height=0.7\textwidth]{img/ui/give-virtual-money.png}
    \end{center}
    \caption{หน้าให้ Virtual Money และแสดงความคิดเห็นเกี่ยวกับโครงการที่เข้าร่วมกิจกรรม}
    \label{fig:guest-project-info}

    \begin{center}
        \includegraphics[height=0.7\textwidth]{img/ui/give-comment.png}
    \end{center}
    \caption{หน้าแสดงความคิดเห็นเกี่ยวกับโครงการที่เข้าร่วมกิจกรรม}
    \label{fig:give-comment}
\end{figure}