\chapter{\ifproject%
      \ifenglish Project Structure and Methodology\else โครงสร้างและขั้นตอนการทำงาน\fi
  \else%
      \ifenglish Project Structure\else โครงสร้างของโครงงาน\fi
  \fi
 }

ในบทนี้จะกล่าวถึงหลักการ และการออกแบบระบบ

\makeatletter

% \renewcommand\section{\@startsection {section}{1}{\z@}%
%                                    {13.5ex \@plus -1ex \@minus -.2ex}%
%                                    {2.3ex \@plus.2ex}%
%                                    {\normalfont\large\bfseries}}

\makeatother
%\vspace{2ex}
% \titleformat{\section}{\normalfont\bfseries}{\thesection}{1em}{}
% \titlespacing*{\section}{0pt}{10ex}{0pt}

\section{หลักการทำงานของระบบ}


% \begin{figure}
% \begin{center}
% \includegraphics{800px-Briny_Beach.jpg}
% \end{center}
% \caption[Poem]{The Walrus and the Carpenter}
% \label{fig:walrus}
% \end{figure}

\subsection{ภาพรวมของระบบ (System Overview)}
\begin{figure}[h]
    \begin{center}
        \includegraphics[width=0.6\textwidth]{800px-Briny_Beach.jpg}
    \end{center}
    \caption[Poem]{The Walrus and the Carpenter}
    \label{fig:walrus}
\end{figure}

ภาพรวมการทำงานของระบบนี้ จะมีส่วนการทำงานหลัก ๆ ดังนี้
\begin{itemize}
    \item ระบบการอัพโหลดและการทำนายรอยโรคในช่องปากโดย AI
    \item ระบบประวัติการอัพโหลดและการทำนายรอยโรคในช่องปากสำหรับแต่ละผู้ใช้งาน
    \item ระบบการวงรอยโรคสำหรับทันตแพทย์ (Annotation)
    \item ระบบการแพทย์ทางไกล (Telemedicine) สำหรับให้ความคิดเห็นระหว่างทันตแพทย์ผู้เชี่ยวชาญกับทันตแพทย์ทั่วไป, อาสาสมัครสาธารณสุขประจำหมู่บ้าน, ทันตบุคลากรและผู้ใช้ทั่วไป
    \item ระบบจัดการผู้ใช้งาน (User Management) และระบบติดตามผู้ใช้งาน (User Tracking)
    \item ระบบจัดการข้อมูล (Data Management)
\end{itemize}

\subsubsection{ระบบการอัพโหลดและการทำนายรอยโรคในช่องปากโดย AI}
ระบบการอัพโหลดและการทำนายรอยโรคในช่องปากโดย AI จะมีการทำงานดังนี้

\subsubsection{ระบบการวงรอยโรคสำหรับทันตแพทย์ (Annotation)}

\subsub



\subsection{โครงสร้างฐานข้อมูล (Database Schema)}
ใส่ ER Diagram ของฐานข้อมูลและอธิบายการทำงาน

\section{User Interface}
ใส่ภาพหน้าจอของระบบ และอธิบาย