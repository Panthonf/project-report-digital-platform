\chapter{\ifproject%
      \ifenglish Project Structure and Methodology\else โครงสร้างและขั้นตอนการทำงาน\fi
  \else%
      \ifenglish Project Structure\else โครงสร้างของโครงงาน\fi
  \fi
 }

ในบทนี้จะกล่าวถึงหลักการ และการออกแบบระบบ

\makeatletter

% \renewcommand\section{\@startsection {section}{1}{\z@}%
%                                    {13.5ex \@plus -1ex \@minus -.2ex}%
%                                    {2.3ex \@plus.2ex}%
%                                    {\normalfont\large\bfseries}}

\makeatother
%\vspace{2ex}
% \titleformat{\section}{\normalfont\bfseries}{\thesection}{1em}{}
% \titlespacing*{\section}{0pt}{10ex}{0pt}

\section{หลักการทำงานของระบบ}


% \begin{figure}
% \begin{center}
% \includegraphics{800px-Briny_Beach.jpg}
% \end{center}
% \caption[Poem]{The Walrus and the Carpenter}
% \label{fig:walrus}
% \end{figure}

\subsection{ภาพรวมของระบบ (System Overview)}
ใส่ Diagram ของระบบและอธิบายการทำงานของระบบ

\subsection{โครงสร้างฐานข้อมูล (Database Schema)}
ใส่ ER Diagram ของฐานข้อมูลและอธิบายการทำงาน

\section{User Interface}
ใส่ภาพหน้าจอของระบบ และอธิบาย