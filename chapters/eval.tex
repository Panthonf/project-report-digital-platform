\chapter{\ifproject%
      \ifenglish Experimentation and Results\else การทดลองและผลลัพธ์\fi
  \else%
      \ifenglish System Evaluation\else การประเมินระบบ\fi
  \fi}

\section{การประเมินความพึงพอใจของผู้ใช้}
เนื่องจากการพัฒนาระบบนี้ พัฒนาเพื่อความต้องการจากผู้ใช้ที่เป็นผู้สร้างงานกิจกรรม (Event Manager) ผู้นำเสนอโครงการ (Presenter) และแขกผู้เข้าร่วมงาน (Guest) ซึ่งเป็นผู้ใช้ที่สำคัญของระบบ จึงได้ทำการประเมินว่าระบบซึ่งเป็นเว็บ-แอพพลิเคชั่นนั้นตรงตามความต้องการของผู้ใช้หรือไม่ โดยการทดสอบกับผู้ใช้จริง ๆ โดยการให้ผู้ใช้ทดลองใช้งานจริง และให้คำแนะนำเพื่อปรับปรุงระบบให้ตรงตามความต้องการของผู้ใช้
ทั้ง 3 กลุ่มผู้ใช้งาน


โดยคณะได้นำเว็บแอพพลิเคชั่นนี้ ไปทดลองใช้จริงในกระบวนวิชา Software Engineering ในวัน Project Demo Day 2/2566 โดยผู้ใช้ ดังนี้
\begin{itemize}
    \item ผู้ใช้ที่เป็นผู้สร้างงานกิจกรรม (Event Manager) ได้ทดลองใช้งานระบบโดยการสร้างงานกิจกรรม และจัดการงานกิจกรรม (1 คน)
    \item ผู้ใช้ที่เป็นผู้นำเสนอโครงการ (Presenter) ได้ทดลองใช้งานระบบโดยการสร้างโครงการ และจัดการโครงการ (14 โครงการ)
    \item ผู้ใช้ที่เป็นแขกผู้เข้าร่วมงาน (Guest) ได้ทดลองใช้งานระบบโดยการเข้าร่วมงาน ให้ Virtual Money และแสดงความคิดเห็นในโครงการต่าง ๆ (5 คน)
\end{itemize}

\section{ผลการประเมินความพึงพอใจของผู้ใช้}
โดยการให้กลุ่มผู้ใช้ตอบแบบสำรวจผ่าน Google Form ประเมินความพอใจในด้านต่าง ๆ โดยแบ่งเป็น 2 ด้าน ได้แก่ การออกแบบและการใช้งาน (Design and Usability) และฟังก์ชันการทำงาน (Functionality)
โดยแบ่งการ ประเมินออกเป็น 5 ระดับ
\begin{itemize}
    \item 1. ไม่พอใจมาก (Very Dissatisfied) - หมายถึง ผู้ใช้งานไม่พอใจมากในข้อคำถามนั้น ๆ
    \item 2. พอใจน้อย (Somewhat Dissatisfied) - หมายถึง ผู้ใช้งานพอใจน้อยในข้อคำถามนั้น ๆ
    \item 3. พอใจปานกลาง (Neutral) - หมายถึง ผู้ใช้งานพอใจปานกลางในข้อคำถามนั้น ๆ
    \item 4. พอใจมาก (Somewhat Satisfied) - หมายถึง ผู้ใช้งานพอใจมากในข้อคำถามนั้น ๆ
    \item 5. พอใจมากที่สุด (Very Satisfied) - หมายถึง ผู้ใช้งานพอใจมากที่สุดในข้อคำถามนั้น ๆ
\end{itemize}

\begin{figure}
    \centering
    \includegraphics[width=0.8\textwidth]{img/form1.png}
    \caption{แบบสำรวจความพึงพอใจของผู้ใช้ในส่วนข้อมูลทั่วไป}
    \label{fig:survey1}
\end{figure}

\begin{figure}
    \centering
    \includegraphics[width=0.8\textwidth]{img/form2.png}
    \caption{แบบสำรวจความพึงพอใจของผู้ใช้ในส่วนการออกแบบและการใช้งาน (Design and Usability)}
    \label{fig:survey2}
\end{figure}

\begin{figure}
    \centering
    \includegraphics[width=0.8\textwidth]{img/form3.png}
    \caption{แบบสำรวจความพึงพอใจของผู้ใช้ในส่วนฟังก์ชันการทำงาน (Functionality)}
    \label{fig:survey3}
\end{figure}

\begin{figure}
    \centering
    \includegraphics[width=0.8\textwidth]{img/form4.png}
    \caption{แบบสำรวจความพึงพอใจของผู้ใช้ในความพึงพอใจโดยรวม}
    \label{fig:survey4}
\end{figure}

\clearpage % force the figure to be placed before the next section
\section{สรุปผลการประเมินความพึงพอใจของผู้ใช้}
จากการประเมินความพึงพอใจของผู้ใช้ พบว่าผู้ใช้งานทั้ง 3 กลุ่ม คือ
ผู้สร้างงานกิจกรรม (Event Manager) ผู้นำเสนอโครงการ (Presenter)
และแขกผู้เข้าร่วมงาน (Guest) มีความพึงพอใจในการใช้งานระบบที่พัฒนาขึ้น
โดยคะแนนเฉลี่ยของการประเมินความพึงพอใจของผู้ใช้ทั้ง 3 กลุ่มโดยรวม คือ 4.2
จาก 5


โดยสามารถแยกตามด้านการประเมินความพึงพอใจของผู้ใช้ได้ดังนี้
\begin{itemize}
    \item การออกแบบและการใช้งาน (Design and Usability) คะแนนเฉลี่ย 4.1 จาก 5
    \item ฟังก์ชันการทำงาน (Functionality) คะแนนเฉลี่ย 4.3 จาก 5
\end{itemize}

และมีข้อเสนอแนะในการปรับปรุงระบบเพื่อให้ตรงตามความต้องการของผู้ใช้ ดังนี้
\begin{itemize}
    \item ผู้สร้างงานกิจกรรม (Event Manager)
          \subitem - ต้องการให้ระบบมีการแจ้งเตือนเมื่อมีโครงการใหม่เพิ่มเข้ามา
          \subitem - ต้องการให้มีการแสดงรายละเอียดโดยรวมของงานกิจกรรม (Dashboard) เช่น จำนวนโครงการที่เข้าร่วม จำนวนแขกที่เข้าร่วม การให้ Virtual Money ในแต่ละโครงการ และอื่น ๆ
    \item ผู้นำเสนอโครงการ (Presenter)
          \subitem - ต้องการให้มีการแจ้งเตือนเมื่อมีผู้มาให้ Virtual Money หรือแสดงความคิดเห็นในโครงการของตน
          \subitem - ต้องการให้ระบบสามารถเพิ่มผู้จัดการโครงการ (Project Manager) คนอื่นในโครงการของตน
    \item แขกผู้เข้าร่วมงาน (Guest)
          \subitem - ต้องการให้การแสดงผลโครงการต่าง ๆ ชัดเจนและสามารถกรองโครงการต่าง ๆ ได้ง่าย เช่น กรองจากหมวดหมู่ อักษรตัวแรก หรือจาก Virtual Money ที่ได้รับ และอื่น ๆ
\end{itemize}
