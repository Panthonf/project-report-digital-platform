\chapter{\ifproject%
      \ifenglish Experimentation and Results\else การทดลองและผลลัพธ์\fi
  \else%
      \ifenglish System Evaluation\else การประเมินระบบ\fi
  \fi}

ในบทนี้จะทดสอบเกี่ยวกับการทำงานในฟังก์ชันหลัก ๆ

\section{การทดลองเกี่ยวกับการทำงานของระบบ}
การประเมินระบบจะประเมินโดยทดสอบกับกลุ่มผู้ใช้งานทั้ง 4 กลุ่ม ได้แก่ ผู้ใช้ทั่วไป, ทันตแพทย์, ทันตบุคลากร และอาสาสมัครสาธารณสุขประจําหมู่บ้าน (อสม.) โดยในการทดสอบระบบจะมีการประเมินผลการทดลองโดยใช้เกณฑ์ต่าง ๆ ดังนี้
\subsection{ผู้ใช้ทั่วไป}
ผู้ใช้ทั่วไปมักมีความต้องการใช้งานระบบที่เรียบง่าย ใช้งานง่าย ไม่ซับซ้อน ดังนั้น ในการทดสอบกับผู้ใช้ทั่วไป ควรเน้นการประเมินปัจจัยต่างๆ เช่น

\begin{itemize}
    \item ความน่าใช้งาน: Ease of use
    \item ความพึงพอใจของผู้ใช้งาน: User satisfaction
    \item ประโยชน์: Benefits


          ตัวอย่างวิธีการทดสอบกับผู้ใช้ทั่วไป ได้แก่
    \item ให้ผู้ใช้ทดสอบระบบและรวบรวมข้อมูลเกี่ยวกับประสบการณ์การใช้งาน เช่น ระยะเวลาในการดำเนินการแต่ละขั้นตอน ความสะดวกในการใช้งาน เป็นต้น
    \item ให้ผู้ใช้ตอบแบบสอบถามเกี่ยวกับความพึงพอใจต่อระบบ เช่น ความง่ายในการใช้งาน ความน่าสนใจของเนื้อหา เป็นต้น
    \item ให้ผู้ใช้ประเมินประโยชน์ที่ได้รับจากระบบ เช่น ช่วยให้ประหยัดเวลา ช่วยให้เข้าใจข้อมูลต่างๆ ได้ง่าย เป็นต้น
\end{itemize}


\subsection{ทันตแพทย์}

ทันตแพทย์มีความต้องการใช้งานระบบที่มีประสิทธิภาพ ถูกต้องแม่นยำและสามารถช่วยในตรวจคัดกรองมะเร็งช่องปากได้อย่างมีประสิทธิภาพ ดังนั้น ในการทดสอบกับทันตแพทย์ ควรเน้นการประเมินปัจจัยต่างๆ เช่น

\begin{itemize}
    \item ความน่าใช้งาน: Ease of use
    \item ความพึงพอใจของผู้ใช้งาน: User satisfaction
    \item ประโยชน์: Benefits
    \item การช่วยในการตรวจคัดกรองมะเร็งช่องปาก: Screening


          ตัวอย่างวิธีการทดสอบกับทันตแพทย์ ได้แก่

    \item ให้ทันตแพทย์ทดสอบระบบภายใต้สถานการณ์จริง เช่น ถ่ายภาพช่องปากของผู้ป่วย และให้ระบบตรวจคัดกรอง และให้ทันตแพทย์ประเมินความถูกต้องแม่นยำของระบบ เป็นต้น
    \item ให้ทันตแพทย์ประเมินประโยชน์ที่ได้รับจากระบบ เช่น ช่วยให้ตรวจคัดกรองมะเร็งช่องปากได้อย่างมีประสิทธิภาพหรือไม่ เป็นต้น

\end{itemize}


\subsection{ทันตบุคลากร}

ทันตบุคลากรมีความต้องการใช้งานระบบที่อำนวยความสะดวกในการทำงาน เช่น การดูประวัติการตรวจคัดกรองมะเร็งช่องปาก การบันทึกข้อมูล การสรุปผลการตรวจคัดกรองมะเร็งช่องปาก ดังนั้น ในการทดสอบกับทันตบุคลากร ควรเน้นการประเมินปัจจัยต่างๆ เช่น

\begin{itemize}
\item ความสะดวกในการใช้งาน: Ease of use
\item ประโยชน์: Benefits


ตัวอย่างวิธีการทดสอบกับทันตบุคลากร ได้แก่

\item ให้ทันตบุคลากรทดสอบระบบและรวบรวมข้อมูลเกี่ยวกับประสบการณ์การใช้งาน เช่น ระยะเวลาในการดำเนินการแต่ละขั้นตอน ความสะดวกในการใช้งาน เป็นต้น
\item ให้ทันตบุคลากรประเมินประโยชน์ที่ได้รับจากระบบ เช่น ระบบช่วยให้ทำงานได้อย่างมีประสิทธิภาพหรือไม่ เป็นต้น
\end{itemize}

\subsection{อาสาสมัครสาธารณสุขประจําหมู่บ้าน (อสม.)}

อสม. มีความต้องการใช้งานระบบที่เข้าใจง่าย ใช้งานสะดวก และสามารถช่วยให้ให้บริการประชาชนได้อย่างมีประสิทธิภาพ ดังนั้น ในการทดสอบกับอสม. ควรเน้นการประเมินปัจจัยต่าง ๆ เช่น
\begin{itemize}
\item ความน่าใช้งาน: Ease of use
\item ความพึงพอใจของผู้ใช้งาน: User satisfaction
\item ประโยชน์: Benefits



ตัวอย่างวิธีการทดสอบกับอสม. ได้แก่

\item ให้อสม.ทดสอบระบบและรวบรวมข้อมูลเกี่ยวกับประสบการณ์การใช้งาน เช่น ระยะเวลาในการดำเนินการแต่ละขั้นตอน ความสะดวกในการใช้งาน เป็นต้น
\item ให้อสม.ตอบแบบสอบถามเกี่ยวกับความพึงพอใจต่อระบบ เช่น ความง่ายในการใช้งาน ความน่าสนใจของเนื้อหา เป็นต้น
\item ให้อสม.ประเมินประโยชน์ที่ได้รับจากระบบ เช่น ระบบช่วยให้ให้บริการประชาชนได้อย่างมีประสิทธิภาพหรือไม่ เป็นต้น
\end{itemize}
ทั้งนี้ ในการทดสอบระบบกับผู้ใช้ทั่วไป, ทันตแพทย์, ทันตบุคลากรและอสม จะพิจารณาจากปัจจัยต่าง ๆ เช่น วัตถุประสงค์ของการประเมิน ขอบเขตของการประเมิน ความพร้อมของระบบ เป็นต้น เพื่อให้ได้ผลการประเมินที่มีประสิทธิภาพ
