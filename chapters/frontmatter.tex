\maketitle
\makesignature

\ifproject
\begin{abstractTH}

ในปัจจุบันปัญหาด้านสุขภาพของประชากรมีแนวโน้มจะสูงขึ้นเรื่อย ๆ ประกอบกับการเข้าสู่สังคมสูงวัยของประชากร ปัญหาสุขภาพจึงเป็นปัญหาที่สำคัญ ซึ่งส่งผลกระทบต่อชีวิตของประชากรโดยส่วนมาก มะเร็งช่องปากเป็นมะเร็งชนิดหนึ่งที่พบมากในกลุ่มประชากรที่มีอายุตั้งแต่ 40 ปีขึ้นไป ที่มีประวัติด้านการสูบบุหรี่ และ
/หรือดื่มแอลกอฮอล์ และเคี้ยวหมาก ซึ่งการตรวจสอบรอยโรคในระยะแรกอาจทำได้ยากโดยทั่วไปและหากปล่อยเป็นระยะเวลานานเกินไปอาจทำให้รอยโรคลุกลามเป็นมะเร็งได้ในที่สุด คณะผู้จัดทำมีความสนใจ
ในเรื่องนี้ จึงได้จัดทำดิจิทัลแพลตฟอร์มเพื่อรองรับระบบปัญญาประดิษฐ์ (AI) เพื่อใช้ในการตรวจสอบและคัดกรองรอยโรคก่อนมะเร็งช่องปากและมะเร็งช่องปาก ที่ใช้ร่วมกับการประเมินจากทันตแพทย์ผู้เชี่ยวชาญ

คณะผู้จัดทำจึงได้นำเสนอการพัฒนาเว็บแอปพลิเคชัน ซึ่งเป็นแพลตฟอร์มดิจิทัลสำหรับรองรับระบบปัญ-ญาประดิษฐ์ (AI) เพื่อตรวจคัดกรองและเฝ้าระวังการเกิดรอยโรคก่อนมะเร็งและมะเร็งช่องปาก (Digital Platform for Detecting and Analyzing Oral Potentially Malignant Disorders and Oral Cancer) โดยกลุ่มผู้ใช้งานของดิจิทัลแพลตฟอร์มนี้จะเป็นทันตแพทย์ทั่วประเทศและประชากรทั่วไปที่มีความสน ใจในการนำดิจิทัลแพลตฟอร์มนี้ไปใช้

คณะผู้จัดทำหวังว่า ดิจิทัลแพลตฟอร์มนี้จะส่งผลให้ทันตแพทย์ทั่วประเทศสามารถตรวจหามะเร็งช่องปากได้อย่างรวดเร็ว และเป็นเครื่องมือหนึ่งที่จะช่วยแก้ไขปัญหาด้านสุขภาพของประชากรโดยเฉพาะมะเร็งช่องปากได้อย่างมีประสิทธิภาพ
\end{abstractTH}

\begin{abstract}

At present, the health problems of the population tend to increase more and more, together with the population entering an aging society. Health problems are therefore important. which affects the lives of the majority of the population Oral cancer is a type of cancer that is most commonly found in people aged 40 and over who have a history of smoking and/or drink alcohol and chew betel nuts. Detecting the Oral Cancer in its early stages may be difficult in general, and if left for too long, it may eventually cause the lesion to develop into cancer. The organizing team is interested.
In this regard, a digital platform has been created to support artificial intelligence (AI) systems for use in examining and screening Oral Potentially Malignant Disorders and Oral Cancer. used in conjunction with an evaluation from an expert dentist.

The team therefore presented the development of a web application. which is a digital platform for supporting the artificial intelligence (AI) system for Detecting and Analyzing Oral Potentially Malignant Disorders and Oral Cancer by a group of The users of this digital platform will be dentists across the country and the general population who are interested in using this digital platform.

The organizing team hopes that this digital platform will allow dentists across the country to quickly detect oral cancer. And it is one tool that will help effectively solve the health problems of the population, especially oral cancer.
\end{abstract}

\iffalse
\begin{dedication}
This document is dedicated to all Chiang Mai University students.

Dedication page is optional.
\end{dedication}
\fi % \iffalse

\begin{acknowledgments}
Your acknowledgments go here. Make sure it sits inside the
\texttt{acknowledgment} environment.

\acksign{2020}{5}{25}
\end{acknowledgments}%
\fi % \ifproject

\contentspage

\ifproject
\figurelistpage

\tablelistpage
\fi % \ifproject

% \abbrlist % this page is optional

% \symlist % this page is optional

% \preface % this section is optional
