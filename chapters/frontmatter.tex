\maketitle
\makesignature

\ifproject
    \begin{abstractTH}

        Gallery Walk คือ การเรียนรู้แบบเดิมชมผลงาน ซึ่งเป็นเทคนิคการเรียนรู้ที่กระตุ้นให้ผู้เรียนแลกเปลี่ยนข้อมูล ความรู้ หรือประสบการณ์ร่วมกันโดยใช้ผลงานที่ผู้เรียนแต่ละกลุ่มร่วมกันสร้างสรรค์ขึ้นเพื่อเป็นสื่อกลางในการถ่ายทอดและใช้กระดานสนทนาในการรับคำติชมและข้อเสนอแนะจากผู้เข้าร่วม เพื่อนำข้อมูลที่ได้รับกลับมาพัฒนาผลงานของกลุ่มหรือเติมเต็มความรู้ความเข้าใจของตัวผู้เรียนเอง

        ในปัจจุบันการจัดกิจกรรมในรูปแบบ Gallery Walk นั้นมีความยุ่งยากในการจัดการ เนื่องจากผู้จัดงานกิจกรรม (Event Manager) จะต้องจัดเตรียมสถานที่ สื่อการนำเสนอ และการจัดการเอกสารต่าง ๆ ที่เกี่ยวข้องกับกิจกรรม เป็นภาระต่อผู้จัดงานกิจกรรม นอกจากนี้ยังไม่สามารถจัดกิจกกรรมในรูปแบบ Hybrid ได้ ในกรณีที่ผู้เข้าร่วมงาน (Guest) ต้องการที่จะเยี่ยมชมผลงาน โครงการต่าง ๆ แบบออนไลน์ เพื่อให้ Virtual Money และแสดงความคิดเห็นต่าง ๆ และส่วนสุดท้ายการแสดงผลหลังสิ้นสุดกิจกรรมไม่สามารถแสดงได้ทันที ทำให้การประกาศผลของผลงานที่ได้รับการประเมินสูงสุดทำได้ล่าช้า

        จากเหตุผลข้างต้นที่กล่าวมานั้น คณะผู้จัดทำจึงได้จัดทำโครงงานเพื่อพัฒนาเว็บแอปพลิเคชันสำหรับการจัดกิจกรรม Gallery Walk ที่สามารถจัดกิจกรรมทั้งในรูปแบบ online และ hybrid ซึ่งสามารถให้ผู้จัดงานกิจกรรมสามารถจัดการกิจกรรมได้ง่ายขึ้น และสามารถให้ผู้เข้าร่วมงานสามารถเยี่ยมชมผลงาน โครงการต่าง ๆ ทั้ง online และ hybrid ซึ่งยังสามารถแสดงความคิดเห็นต่าง ๆ และส่วนสุดท้ายสามารถแสดงผลได้ทันทีว่าผลงานใดได้รับการประเมินสูงสุดหลังเสร็จสิ้นงานกิจกรรม
    \end{abstractTH}

    \begin{abstract}

        Gallery Walk is the traditional way of learning and viewing works. This is a learning technique that encourages students to exchange information. Shared knowledge or experience by using the work that each group of learners jointly created as a medium for transferring and using a discussion board as a medium for receiving feedback and suggestions from others. To use that feedback to develop the group's work or complete the learners' own knowledge and understanding.

        At present, organizing an activity in the form of a gallery walk is difficult to manage. This is because the event organizer (the event manager) must prepare the venue. communication media and managing various documents related to activities. It is a burden on event organizers. In addition, activities cannot be organized in a hybrid format in cases where attendees (guests) would like to visit the various projects online in order to provide virtual money and express various opinions. And the last part is that it cannot immediately show which work received the highest evaluation after the event has been completed.

        From the above-mentioned The organizing team therefore created a project to develop a web application for organizing Gallery Walk activities in a hybrid format, which can allow event organizers to manage their activities more easily. And it can allow attendees to visit the various projects online and express various opinions, and the last part can immediately show which work received the highest evaluation after the event has finished.
    \end{abstract}

    \iffalse
        \begin{dedication}
            This document is dedicated to all Chiang Mai University students.

            Dedication page is optional.
        \end{dedication}
    \fi % \iffalse

    \begin{acknowledgments}
        
        
        โครงงานนี้ได้เสร็จสมบูรณ์ด้วยความกรุณาอย่างยิ่งจาก ผศ.โดม โพธิกานนท์ ผู้ที่ได้มอบเวลาและความกรุณาเพื่อเป็นที่ปรึกษาในโครงงานนี้ ได้ให้คำแนะนำที่มีคุณค่าและแนวทางที่ช่วยให้โครงงานเกิดขึ้นอย่างสมบูรณ์และประสบความสำเร็จ
        
        ไม่เพียงแต่นั้นเท่านั้น ขอขอบคุณอย่างสูงสำหรับคำปรึกษาจาก อ.ดร.ชินวัตร อิศราดิสัยกุล และ ผศ.ดร.ธนาทิพย์ จันทร์คง ที่ได้มอบคำแนะนำและเสนอแนวทางที่มีประสิทธิภาพ ทำให้โครงงานเกิดประสิทธิภาพมากขึ้น
        
        นอกจากนี้ เราต้องขอบคุณเพื่อนร่วมทีมทุกคนที่ให้การสนับสนุนและกำลังใจ และขอบคุณครอบครัวที่เป็นกำลังใจสำคัญในการผ่านช่วงเวลาที่ท้าทายไปด้วยกัน ขอบคุณทุกท่านที่เคยให้คำแนะนำและความช่วยเหลือ โดยที่ไม่ระบุนาม ทุกคำปรึกษาและการสนับสนุนที่มีค่ามากสำหรับเรา
        
        หากหนังสือโครงงานเล่มนี้มีข้อผิดพลาดประการใดผู้จัดทําขอน้อมรับด้วย ความยินดียิ่งและขออภัยมา ณ ที่นี้ด้วย
        \acksign{2024}{3}{20}
    \end{acknowledgments}%
\fi % \ifproject

\contentspage

\ifproject
    \figurelistpage

    \tablelistpage
\fi % \ifproject

% \abbrlist % this page is optional

% \symlist % this page is optional

% \preface % this section is optional
