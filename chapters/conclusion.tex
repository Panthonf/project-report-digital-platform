\chapter{\ifenglish Conclusions and Discussions\else บทสรุปและข้อเสนอแนะ\fi}

\section{\ifenglish Conclusions\else สรุปผล\fi}
การทําโครงงานนี้สามารถพัฒนาเว็บแอปพลิเคชั่นที่สามารถใช้งานตามความต้องการของผู้ใช้งานได้
โดยผู้ใช้งานทั้ง 3 กลุ่ม คือ
\begin{itemize}
      \item ผู้ใช้ที่เป็นผู้สร้างงานกิจกรรม (Event Manager)
            สามารถสร้างงานกิจกรรม และจัดการงานกิจกรรมได้
      \item ผู้ใช้ที่เป็นผู้นำเสนอโครงการ (Presenter)
            สามารถสร้างโครงการ และจัดการโครงการได้
      \item ผู้ใช้ที่เป็นแขกผู้เข้าร่วมงาน (Guest)
            สามารถดูงานกิจกรรมและโครงการต่าง ๆ และให้ Virtual Money และแสดงความคิดเห็นในโครงการต่าง ๆ ได้
\end{itemize}
\section{\ifenglish Challenges\else ปัญหาที่พบและแนวทางการแก้ไข\fi}
ในการทำโครงงานนี้ พบว่าเกิดปัญหาหลัก ๆ ดังนี้

\begin{itemize}
      \item การทำงานร่วมกันของทีม
            ทีมมีคนทำงานร่วมกัน 3 คน แต่เนื่องจากการประชุมและการทำงานร่วมกันไม่สม่ำเสมอ ทำให้การทำงานมีความล่าช้าอยู่บ้าง
      \item การทดสอบระบบ
            การทดสอบระบบระหว่างการพัฒนาไม่ครอบคลุม ทำให้มีบางส่วนของระบบที่ไม่ทำงานอย่างที่ควรจะเป็น
      \item การแสดงผลข้อมูล
            การแสดงผลข้อมูลตามกลุ่มผู้ใช้งานไม่ครอบคลุม ทำให้ผู้ใช้งานบางกลุ่มไม่สามารถใช้งานระบบได้อย่างที่ควรจะเป็น
\end{itemize}

\section{\ifenglish%
        Suggestions and further improvements
  \else%
        ข้อเสนอแนะและแนวทางการพัฒนาต่อ
  \fi
 }

ข้อเสนอแนะเพื่อพัฒนาโครงงานนี้ต่อไป มีดังนี้


ข้อเสนอที่ได้รับจากแบบสำรวจความพึงพอใจของผู้ใช้งานและผู้พัฒนาระบบ ได้แก่
\begin{itemize}
      \item พัฒนาส่วนแดชบฮร์ดสำหรับผู้ใช้งานที่เป็นผู้สร้างงานกิจกรรม (Event Manager) ให้มีการแสดงผลข้อมูลที่ครอบคลุมมากขึ้นในแต่ละกิจกรรม ซึ่งอาจประกอบไปด้วย จำนวนผู้เข้าร่วมงาน จำนวนโครงการที่เข้าร่วม จำนวน Virtual Money ที่แต่ละโครงการได้รับ แขกผู้เข้าร่วมงานให้ Virtual Money และแสดงความคิดเห็นในโครงการต่าง ๆ
      \item พัฒนาการแจ้งเตือนให้ผู้ใช้ที่เป็นผู้นำเสนอโครงการ (Presenter) ให้มีการแจ้งเตือนเมื่อโครงการของตนได้รับ Virtual Money หรือมีความคิดเห็นใหม่ ๆ
      \item ในส่วนของการเพิ่มโครงการเข้ามาในงานกิจกรรม ให้มีการตรวจสอบและอนุมัติ (Approve) โครงการจากผู้สร้างงานกิจกรรมก่อนที่จะแสดงผลในงานกิจกรรม
\end{itemize}

ข้อเสนอแนะที่ได้ผู้เป็นแรงบันดาลใจในการพัฒนา (inspiration)


จากอาจารย์บุลวิชช์ ช่วยชูวงศ์ (พี่นุ) ผู้เชี่ยวชาญด้าน Design Thinking
โดยคณะผู้จัดทำได้นำแพลทฟอร์มนี้ไปสาธิตการทำงานทั้ง 3 กลุ่มผู้ใช้งาน และได้รับคำแนะนำในการพัฒนาและเเนวทางที่จะพัฒนาโปรเเกรมของเราให้ดียิ่งขึ้น โดยได้เล็งเห็นว่าแพลทฟอร์มนี้จะช่วยตอบโจทย์กลุ่มผู็ใช้งานที่ต้องการจัดงานกิจกรรมและมี KPI (Key Performance Indicator) คือจำนวนผู้เข้าร่วมชมกิจกรรมได้ เช่น การที่สสส. ต้องจัดกิจกรรมโชว์ผลงานโครงการส่งเสริมสุขภาพภายใต้งบของ สสส. จำนวน 50 โครงการ โดยมีเป้าว่าต้องมีผู้เข้าชมไม่ต่ำกว่า 2,000 คน สสส.สามารถจัดงานกิจกรรมแบบ on-site เชิญชวนคนมาดูงาน และ
จัดกิจกรรมแบบ online ขนานกันไป การจัด on-site ก็ไม่จำเป็นต้องรองรับคน 2,000 คนก็ได้ อาจจะรับได้ 200 คน แล้วก็เชิญผู้ชมออนไลน์เข้าอีกได้พันกว่าคน ด้วยระบบของเรา ซึ่งจะช่วยอำนวยความสะดวกในเรื่องของการจัดงานเป็นอย่างมาก เเละเนื่องจากหลาย ๆ ท่านก็ได้เล็งเห็นถึงความเป็นไปได้ในการที่จะพัฒนาโครงการนี้ต่อ โดยมีสิ่งที่ได้เเนะนำมาเพิ่มเติมดังนี้
\begin{itemize}
      \item Investment return \\
            โดยทางอาจารย์บุลวิชช์ ได้มองว่าเป็นคล้าย ๆ กับการลงทุน คือ สำหรับ Guest หรือนักลงทุนจำลองที่มาทำการลงทุน ก็จะต้องได้รับผลตอบแทนการลงทุน เพื่อกระตุ้นให้ guest เลือกลงทุนในโปรเจคที่มีศักยภาพในการพัฒนาได้จริง สร้างผลกำไรได้จริง ถ้าโครงการที่ Guest เลือกได้รับเงินระดมทุนเยอะในอันดับต้น ๆ ก็จะมี Reward กลับไปให้ เช่น เป็นจำนวนหุ้น(จำลอง) เงินปันผล เป็นต้น
            โดย ฟีเจอร์นี้ต้องทำการปรึกษากับผู้ใช้ระบบโดยละเอียดอีกครั้งว่าควรเป็นรีวอร์ด (Reward) ให้นักลงทุนในรูปแบบใด
      \item Required Feedback \\
            คำถาม Feedback ที่จะเจาะประเด็นต่าง ๆ โดยให้ผู้จัดงานกิจกรรมหรือผู้นำเสนอโครงการสามารถตั้งคำถามเก็บข้อมูลจากผู้เข้าร่วมกิจกรรมได้ตรงกับเป้าหมายของกิจกรรมหรือโครงการนั้น ๆ มากยิ่งขึ้น
      \item Reveal The Guest \\
            ผู้จัดงานกิจกรรม (Event manager) สามารถที่จะระบุได้ว่าอยากจะให้กิจกรรมนั้น ๆ ซ่อนหรือเปิดเผยชื่อคนที่ให้ Virtual Money หรือชื่อคนที่มาแสดงความคิดเห็นได้ เนื่องจากบางโครงการที่เกี่ยวข้องกับการระดมทุนก็จำเป็นที่จะต้องเปิดเผยผู้ที่มาร่วมระดมทุน
      \item Comment Platform \\
            เนื่องจากระบบของเรามีการเเสดงความคิดเห็นเป็นจุดเด่น เราจึงต้องพัฒนาระบบนี้ให้รองรับการใช้งานเป็นแพลตฟอร์มสำหรับการสำรวจ เช่น ทดลองผลิตภัณฑ์ใหม่
            ซึ่งมีตัวอย่างคร่าว ๆ ตามที่อาจารย์บุลวิชช์ได้เเนะนำมา เช่น ซิซเล่อร์ ต้องการสร้างเมนูใหม่ แล้วอยากรู้ว่าจานนี้คนจะยอมจ่ายในราคาเท่าไหร่
            \begin{itemize}
                  \item โดยกำหนดให้ร้านเป็น Event Manager แล้วสร้างเมนูอาหารไว้ในอีเวนท์

                  \item ให้ลูกค้าเข้ามาเป็น Guest กำหนด Virtual money ให้ xxx unit

                  \item ให้ลูกค้าทดลองกินเมนูใหม่

                  \item ให้ลูกค้าให้ Virtual Money ในราคาที่ลูกค้าคิดว่าเหมาะสม

                  \item มีรีวอร์ด (Reward) กลับไปให้ลูกค้า (ฟีเจอร์ตอบแทนการลงทุนด้านบน) แต่กรณีนี้ รีวอร์ดจะไม่เกี่ยวกับเลือกโครงการ Top 3 แต่ให้รีวอร์ดกับลูกค้าทุกคนที่เข้าร่วมงานกิจกรรม
            \end{itemize}
\end{itemize}