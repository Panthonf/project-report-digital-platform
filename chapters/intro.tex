\chapter{\ifenglish Introduction\else บทนำ\fi}

\section{\ifenglish Project rationale\else ที่มาของโครงงาน\fi}

โครงการ Gallery Walk มีความหมายในการให้ผู้เรียนมีโอกาสแลกเปลี่ยนความรู้และประสบการณ์ผ่านผลงานที่สร้างขึ้นเป็นสื่อกลาง และการใช้กระดานสนทนาเพื่อรับคำแนะนำและคำชมเชยจากผู้อื่น เพื่อให้สามารถพัฒนาผลงานหรือเพิ่มเติมความรู้แก่ตนเองได้

การจัดกิจกรรมแบบ Gallery Walk ในปัจจุบันมีความยุ่งยากในการจัดการ เนื่องจากต้องมีการจัดเตรียมสถานที่ สื่อการสื่อสาร และการจัดการเอกสารต่าง ๆ ซึ่งเป็นภาระต่อผู้จัดงาน นอกจากนี้ยังไม่สามารถจัดกิจกรรมแบบ Hybrid ได้ โดยเฉพาะในกรณีที่ผู้เข้าร่วมต้องการที่จะเยี่ยมชมผลงานแบบออนไลน์และให้คำแนะนำ

ดังนั้น เพื่อแก้ไขปัญหาดังกล่าว ทีมผู้จัดงานได้พัฒนาเว็บแอปพลิเคชันสำหรับการจัดกิจกรรม Gallery Walk ในรูปแบบ Hybrid เพื่อให้การจัดการกิจกรรมเป็นไปอย่างมีประสิทธิภาพ และเพื่อให้ผู้เข้าร่วมสามารถเยี่ยมชมผลงานและให้ความเห็นออนไลน์ได้ นอกจากนี้ยังสามารถแสดงผลการประเมินได้ทันทีหลังจากเสร็จสิ้นกิจกรรม

\section{\ifenglish Objectives\else วัตถุประสงค์ของโครงงาน\fi}
\begin{enumerate}
    \item เพื่อให้การจัดกิจกรรมของผู้จัดกิจกรรมสะดวกและมีประสิทธิภาพมากขึ้น
    \item เพื่อสร้างช่องทางสําหรับการดําเนินกิจกรรมทางออนไลน์ให้กับผู้เข้าร่วมกิจกรรม สามารถจัดกิจกรรมในรูปแบบ Hybrid ได้
    \item การประเมินที่เน้นการมีส่วนร่วมและรับความคิดเห็น ซึ่งส่งเสริมการพัฒนาผลงานของผู้นำเสนอโครงการ
\end{enumerate}

\section{\ifenglish Project scope\else ขอบเขตของโครงงาน\fi}

\subsection{\ifenglish Hardware scope\else ขอบเขตด้านฮาร์ดแวร์\fi}
โครงการนี้ต้องการฮาร์ดแวร์ต่อไปนี้ จึงจะสามารถใช้งานได้อย่างมีประสิทธิภาพ
\begin{itemize}
    \item คอมพิวเตอร์ส่วนบุคคลหรือโทรศัพท์มือถือที่สามารถใช้งานเว็บเบราว์เซอร์ได้
\end{itemize}

\subsection{\ifenglish Software scope\else ขอบเขตด้านซอฟต์แวร์\fi}
โครงการนี้ต้องการซอฟต์แวร์ต่อไปนี้ จึงจะสามารถใช้งานได้อย่างมีประสิทธิภาพ
\begin{itemize}
    \item สามารถใช้งานเว็บไซต์บนระบบปฏิบัติการทั่วไปได้ เช่น Windows, macOS, Linux, Android, iOS และอื่น ๆ
\end{itemize}

\section{\ifenglish Expected outcomes\else ประโยชน์ที่ได้รับ\fi}
ผู้ใช้งาน
\begin{itemize}
    \item สามารถใช้งานเว็บแอปพลิเคชันเพื่อตรวจคัดกรองและเฝ้าระวังการเกิดรอยโรคก่อนมะเร็งและมะเร็งช่องปากได้
    \item สามารถเข้าถึงการรักษาทางการแพทย์ได้อย่างรวดเร็ว หลังจากที่ผู้ใช้งานได้รับการตรวจคัดกรองและเฝ้าระวังการเกิดรอยโรคก่อนมะเร็งและมะเร็งช่องปากโดยเว็บแอปพลิเคชัน
\end{itemize}
ผู้พัฒนา
\begin{itemize}
    \item ได้รับความรู้และความเข้าใจในการพัฒนาเว็บแอปพลิเคชันเพื่อรองรับระบบปัญญาประดิษฐ์ (AI)
    \item ได้ฝึกทักษะในการพัฒนาเว็บแอปพลิเคชันเพื่อรองรับระบบปัญญาประดิษฐ์ (AI)
    \item ได้ฝึกทักษะในการทำงานเป็นทีมและทักษะในการวิเคราะห์และแก้ไขปัญหาที่อาจเกิดขึ้นในการพัฒนา
\end{itemize}

\section{\ifenglish Technology and tools\else เทคโนโลยีและเครื่องมือที่ใช้\fi}

\subsection{\ifenglish Hardware technology\else เทคโนโลยีด้านฮาร์ดแวร์\fi}

\subsection{\ifenglish Software technology\else เทคโนโลยีด้านซอฟต์แวร์\fi}
\begin{itemize}
    \item ภาษาโปรแกรมมิ่ง: JavaScript, Python, HTML, CSS
    \item ฐานข้อมูล: MySQL
    \item เครื่องมือและเทคโนโลยี: NextJS, Tailwind CSS, Git, GitHub, Google Cloud Platform
\end{itemize}

\section{\ifenglish Project plan\else แผนการดำเนินงาน\fi}

\begin{plan}{6}{2023}{2}{2024}
    \planitem{6}{2023}{1}{2024}{ศึกษาค้นคว้าเกี่ยวกับเทคโนโลยีที่เกี่ยวข้อง}
    \planitem{9}{2023}{10}{2023}{ออกแบบ UI/UX ของเว็บแอปพลิเคชัน}
    \planitem{11}{2023}{1}{2024}{พัฒนาเว็บแอปพลิเคชัน}
    \planitem{1}{2024}{2}{2024}{ทดสอบและปรับปรุงเว็บแอปพลิเคชัน}
\end{plan}

\section{\ifenglish Roles and responsibilities\else บทบาทและความรับผิดชอบ\fi}
มีหน้าที่และความรับผิดชอบ ดังนี้


นายญาณาธิป ภู่สว่าง รหัสนักศึกษา 630612097 รับผิดชอบในการศึกษาค้นคว้าเทคโนโลยีที่เกี่ยวข้อง, ออกแบบโครงสร้างของเว็บแอปพลิเคชันและพัฒนาเว็บแอปพลิเคชัน


นายปัณฑ์ธร กันทรัพย์ รหัสนักศึกษา 630612105 รับผิดชอบในการศึกษาค้นคว้าเทคโนโลยีที่เกี่ยวข้อง, ออกแบบโครงสร้างของเว็บแอปพลิเคชันและพัฒนาเว็บแอปพลิเคชัน



% \section{\ifenglish%
%       Impacts of this project on society, health, safety, legal, and cultural issues
%   \else%
%       ผลกระทบด้านสังคม สุขภาพ ความปลอดภัย กฎหมาย และวัฒนธรรม
%   \fi}

% แนวทางและโยชน์ในการประยุกต์ใช้งานโครงงานกับงานในด้านอื่นๆ รวมถึงผลกระทบในด้านสังคมและสิ่งแวดล้อมจากการใช้ความรู้ทางวิศวกรรมที่ได้
