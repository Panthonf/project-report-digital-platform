\chapter{\ifenglish Introduction\else บทนำ\fi}

\section{\ifenglish Project rationale\else ที่มาของโครงงาน\fi}

ในปัจจุบันปัญหาด้านสุขภาพของประชากรมีแนวโน้มจะสูงขึ้นเรื่อย ๆ ประกอบกับการเข้าสู่สังคมสูงวัยของประชากร ปัญหาสุขภาพจึงเป็นปัญหาที่สำคัญ ซึ่งส่งผลกระทบต่อชีวิตของประชากรโดยส่วนมาก มะเร็งช่องปากเป็นมะเร็งชนิดหนึ่งที่พบมากในกลุ่มประชากรที่มีอายุตั้งแต่ 40 ปีขึ้นไป ที่มีประวัติด้านการสูบบุหรี่ และ
/หรือดื่มแอลกอฮอล์ และเคี้ยวหมาก ซึ่งการตรวจสอบรอยโรคในระยะแรกอาจทำได้ยากโดยทั่วไปและหากปล่อยเป็นระยะเวลานานเกินไปอาจทำให้รอยโรคลุกลามเป็นมะเร็งได้ในที่สุด คณะผู้จัดทำมีความสนใจ
ในเรื่องนี้ จึงได้จัดทำดิจิทัลแพลตฟอร์มเพื่อรองรับระบบปัญญาประดิษฐ์ (AI) เพื่อใช้ในการตรวจสอบและคัดกรองรอยโรคก่อนมะเร็งช่องปากและมะเร็งช่องปาก ที่ใช้ร่วมกับการประเมินจากทันตแพทย์ผู้เชี่ยวชาญ
    
คณะผู้จัดทำจึงได้นำเสนอการพัฒนาเว็บแอปพลิเคชัน ซึ่งเป็นแพลตฟอร์มดิจิทัลสำหรับรองรับระบบปัญ-ญาประดิษฐ์ (AI) เพื่อตรวจคัดกรองและเฝ้าระวังการเกิดรอยโรคก่อนมะเร็งและมะเร็งช่องปาก (Digital Platform for Detecting and Analyzing Oral Potentially Malignant Disorders and Oral Cancer) โดยกลุ่มผู้ใช้งานของดิจิทัลแพลตฟอร์มนี้จะเป็นทันตแพทย์ทั่วประเทศและประชากรทั่วไปที่มีความสน ใจในการนำดิจิทัลแพลตฟอร์มนี้ไปใช้
    
คณะผู้จัดทำหวังว่า ดิจิทัลแพลตฟอร์มนี้จะส่งผลให้ทันตแพทย์ทั่วประเทศสามารถตรวจหามะเร็งช่องปากได้อย่างรวดเร็ว และเป็นเครื่องมือหนึ่งที่จะช่วยแก้ไขปัญหาด้านสุขภาพของประชากรโดยเฉพาะมะเร็งช่องปากได้อย่างมีประสิทธิภาพ    


\section{\ifenglish Objectives\else วัตถุประสงค์ของโครงงาน\fi}
\begin{enumerate}
    \item
\end{enumerate}

\section{\ifenglish Project scope\else ขอบเขตของโครงงาน\fi}

\subsection{\ifenglish Hardware scope\else ขอบเขตด้านฮาร์ดแวร์\fi}

\subsection{\ifenglish Software scope\else ขอบเขตด้านซอฟต์แวร์\fi}

\section{\ifenglish Expected outcomes\else ประโยชน์ที่ได้รับ\fi}

\section{\ifenglish Technology and tools\else เทคโนโลยีและเครื่องมือที่ใช้\fi}

\subsection{\ifenglish Hardware technology\else เทคโนโลยีด้านฮาร์ดแวร์\fi}

\subsection{\ifenglish Software technology\else เทคโนโลยีด้านซอฟต์แวร์\fi}

\section{\ifenglish Project plan\else แผนการดำเนินงาน\fi}

\begin{plan}{6}{2020}{2}{2021}
    \planitem{7}{2020}{8}{2020}{ศึกษาค้นคว้า}
    \planitem{8}{2020}{1}{2021}{ชิล}
    \planitem{2}{2021}{2}{2021}{เผา}
    \planitem{12}{2019}{1}{2022}{ทดสอบ}
\end{plan}

\section{\ifenglish Roles and responsibilities\else บทบาทและความรับผิดชอบ\fi}
อธิบายว่าในการทำงาน นศ. มีการกำหนดบทบาทและแบ่งหน้าที่งานอย่างไรในการทำงาน จำเป็นต้องใช้ความรู้ใดในการทำงานบ้าง

\section{\ifenglish%
Impacts of this project on society, health, safety, legal, and cultural issues
\else%
ผลกระทบด้านสังคม สุขภาพ ความปลอดภัย กฎหมาย และวัฒนธรรม
\fi}

แนวทางและโยชน์ในการประยุกต์ใช้งานโครงงานกับงานในด้านอื่นๆ รวมถึงผลกระทบในด้านสังคมและสิ่งแวดล้อมจากการใช้ความรู้ทางวิศวกรรมที่ได้
