% \chapter{The first appendix}

% Text for the first appendix goes here.

% \section{Appendix section}

% Text for a section in the first appendix goes here.

% test ทดสอบฟอนต์ serif ภาษาไทย

% \textsf{test ทดสอบฟอนต์ sans serif ภาษาไทย}

% \verb+test ทดสอบฟอนต์ teletype ภาษาไทย+

% \texttt{test ทดสอบฟอนต์ teletype ภาษาไทย}

% \textbf{ตัวหนา serif ภาษาไทย \textsf{sans serif ภาษาไทย} \texttt{teletype ภาษาไทย}}

% \textit{ตัวเอียง serif ภาษาไทย \textsf{sans serif ภาษาไทย} \texttt{teletype ภาษาไทย}}

% \textbf{\textit{ตัวหนาเอียง serif ภาษาไทย \textsf{sans serif ภาษาไทย} \texttt{teletype ภาษาไทย}}}

% \url{https://www.example.com/test_ทดสอบ_url}

\chapter{\ifenglish Manual\else คู่มือการติดตั้ง (สำหรับการพัฒนา)\fi}
การพัฒนาระบบของแบ่งออกเป็น 2 ส่วน คือ ฝั่งหน้าบ้าน (Front-End) และฝั่งหลังบ้าน (Back-End) จึงต้องทำการติดตั้งแยกกัน ดังนี้

\section{\ifenglish Front-End\else ฝั่งหน้าบ้าน (Front-End)\fi}
\begin{itemize}
    \item สิ่งที่เครื่องสำหรับติดตั้งต้องมีคือ Node.js และ npm โดยสามารถดาวน์โหลดได้จาก \url{https://nodejs.org/en/download/}
    \item Download source code หรือ Clone ได้จาก GitHub \url{https://github.com/Panthonf/frontend-gallery-walk.git}
    \item ทำการเปิด Terminal หรือ Command Prompt และเข้าไปที่โฟลเดอร์ที่เก็บ source code และทำการติดตั้งโดยใช้คำสั่ง \texttt{npm install} เพื่อทำการติดตั้ง package ที่จำเป็น
    \item เพิ่ม .env ไฟล์เพื่อกำหนดค่าต่าง ๆ ได้แก่
          \begin{itemize}
              \item \texttt{VITE\_BACKEND\_ENDPOINT} คือ http://localhost:8080/api/login/google ที่ใช้ในการเชื่อมต่อกับฝั่งหลังบ้าน
              \item \texttt{VITE\_CHECK\_LOGIN} คือ http://localhost:8080/api/isLoggedIn ที่ใช้ในการเชื่อมต่อกับฝั่งหลังบ้าน
              \item \texttt{VITE\_BASE\_ENDPOINTMENT} คือ http://localhost:8080/api ที่ใช้ในการเชื่อมต่อกับฝั่งหลังบ้าน
              \item \texttt{VITE\_FRONTEND\_ENDPOINT} คือ http://localhost:3000 ที่ใช้ในการเชื่อมต่อกับฝั่งหน้าบ้าน
          \end{itemize}
    \item หลังจากติดตั้งเสร็จสิ้น สามารถทำการรันโปรแกรมได้โดยใช้คำสั่ง \texttt{npm run dev} และเข้าไปที่ \url{http://localhost:3000} ในเว็บเบราว์เซอร์ จะพบกับหน้าเว็บที่ใช้ในการใช้งาน
\end{itemize}

\section{\ifenglish Back-End\else ฝั่งหลังบ้าน (Back-End)\fi}
\begin{itemize}
    \item สิ่งที่เครื่องสำหรับติดตั้งต้องมีคือ Node.js และ npm โดยสามารถดาวน์โหลดได้จาก \url{https://nodejs.org/en/download/}
    \item Download source code หรือ Clone ได้จาก GitHub \url{https://github.com/Panthonf/backend-gallery-walk.git}
    \item ทำการเปิด Terminal หรือ Command Prompt และเข้าไปที่โฟลเดอร์ที่เก็บ source code และทำการติดตั้งโดยใช้คำสั่ง \texttt{npm install} เพื่อทำการติดตั้ง package ที่จำเป็น
    \item เพิ่ม .env ไฟล์เพื่อกำหนดค่าต่าง ๆ ได้แก่
          \begin{itemize}
              \item \texttt{PORT} คือ 8080 ที่ใช้ในการเชื่อมต่อกับฝั่งหน้าบ้าน
              \item \texttt{CALLBACK\_URI} คือ http://localhost:8080/api/login/google/callback ที่ใช้ในการเชื่อมต่อกับฝั่งหน้าบ้าน
              \item \texttt{DATABASE\_URL} คือ Connection String ของ PostgreSQL
              \item \texttt{SECRET\_KEY} คือ คีย์สำหรับการเข้ารหัส Token
              \item \texttt{NODE\_ENV} คือ development สำหรับการพัฒนา และ production สำหรับการใช้งานจริง
              \item \texttt{FRONTEND\_URL} คือ http://localhost:3000
              \item \texttt{MINIO\_ENDPOINT} คือ Minio Server
              \item \texttt{MINIO\_URL} คือ Minio Server
              \item \texttt{MINIO\_ACCESSKEY} คือ Minio Access Key
              \item \texttt{MINIO\_SECRETKEY} คือ Minio Secret Key
              \item \texttt{MINIO\_PORT} คือ 443
              \item \texttt{MINIO\_USESSL} คือ true
              \item \texttt{CALLBACK\_URI\_GUEST} คือ http://localhost:8080/api/guests/login/google/callback
          \end{itemize}
    \item ทำการสร้าง Database โดยใช้คำสั่ง \texttt{npx prisma migrate dev --name init} จะทำการสร้าง Database และ Table ตามที่กำหนดไว้ในไฟล์ \texttt{schema.prisma}
    \item หลังจากติดตั้งเสร็จสิ้น สามารถทำการรันโปรแกรมได้โดยใช้คำสั่ง \texttt{npm run dev} และสามารถเรียกใช้งาน API ผ่าน Postman หรือเว็บเบราว์เซอร์ได้
\end{itemize}


\chapter{\ifenglish User Manual\else คู่มือการใช้งาน\fi}
สามารถดูวิธีการใช้งานได้จากวิดีโอที่อัพโหลดไว้ที่ \url{https://www.youtube.com/watch?v=} 

