\documentclass[final]{cpecmu}

%% This is a sample document demonstrating how to use the CPECMU
%% project template. If you are having trouble, see "cpecmu.pdf" for
%% documentation.

\projectNo{P809-2}
\acadyear{2023}

\titleTH{แพลตฟอร์มการแลกเปลี่ยนความคิดสร้างสรรค์ดิจิทัล: การเสริมสร้างการเรียนรู้ผ่าน Gallery Walk}
\titleEN{Digital Creativity Exchange Platform: Enhancing Learning through Gallery Walk}

\author{นายญาณาธิป ภู่สว่าง}{Yanatip Bhoosawang}{630612097}
\author{นางสาวณัฐวรรณ เรียบเรียง}{Nuttawan Reabreang}{630612099}
\author{นายปัณฑ์ธร กันทรัพย์}{Panthon Kansap}{630612105}

\cpeadvisor{dome}
\cpecommittee{chinawat}
\cpecommittee{thanatip}
% \committee{รศ.ดร.\,นิพนธ์ ธีรอำพน}{Assoc.\,Prof.\,Nipon Theera-Umpon, Ph.D.}

%% Some possible packages to include:
\usepackage[final]{graphicx} % for including graphics
\usepackage{cite}

%% Add bookmarks and hyperlinks in the document.
\PassOptionsToPackage{hyphens}{url}
\usepackage[colorlinks=true,allcolors=Blue4,citecolor=red,linktoc=all]{hyperref}
\def\UrlLeft#1\UrlRight{$#1$}

%% Needed just by this example, but maybe not by most reports
\usepackage{afterpage} % for outputting
\usepackage{pdflscape} % for landscape figures and tables. 

%% Some other useful packages. Look these up to find out how to use
%% them.
% \usepackage{natbib}    % for author-year citation styles
% \usepackage{txfonts}
% \usepackage{appendix}  % for appendices on a per-chapter basis
% \usepackage{xtab}      % for tables that go over multiple pages
% \usepackage{subfigure} % for subfigures within a figure
% \usepackage{pstricks,pdftricks} % for access to special PostScript and PDF commands
% \usepackage{nomencl}   % if you have a list of abbreviations

%% if you're having problems with overfull boxes, you may need to increase
%% the tolerance to 9999
% \tolerance=9999

% \bibliographystyle{plain}
% \bibliographystyle{IEEEbib}
\bibliographystyle{IEEEtran}


% \renewcommand{\topfraction}{0.85}
% \renewcommand{\textfraction}{0.1}
% \renewcommand{\floatpagefraction}{0.75}

%% Example for glossary entry
%% Need to use glossary option
%% See glossaries package for complete documentation.
\ifglossary
  \newglossaryentry{lorem ipsum}{
    name=lorem ipsum,
    description={derived from Latin dolorem ipsum, translated as ``pain itself''}
  }
\fi

%% Uncomment this command to preview only specified LaTeX file(s)
%% imported with \include command below.
%% Any other file imported via \include but not specified here will not
%% be previewed.
%% Useful if your report is large, as you might not want to build
%% the entire file when editing a certain part of your report.
% \includeonly{chapters/intro,chapters/background}

\begin{document}
\maketitle
\makesignature

\ifproject
    \begin{abstractTH}

        Gallery Walk คือ การเรียนรู้แบบเดิมชมผลงาน ซึ่งเป็นเทคนิคการเรียนรู้ที่กระตุ้นให้ผู้เรียนแลกเปลี่ยนข้อมูล ความรู้ หรือประสบการณ์ร่วมกันโดยใช้ผลงานที่ผู้เรียนแต่ละกลุ่มร่วมกันสร้างสรรค์ขึ้นเพื่อเป็นสื่อกลางในการถ่ายทอดและใช้กระดานสนทนาในการรับคำติชมและข้อเสนอแนะจากผู้เข้าร่วม เพื่อนำข้อมูลที่ได้รับกลับมาพัฒนาผลงานของกลุ่มหรือเติมเต็มความรู้ความเข้าใจของตัวผู้เรียนเอง

        ในปัจจุบันการจัดกิจกรรมในรูปแบบ Gallery Walk นั้นมีความยุ่งยากในการจัดการ เนื่องจากผู้จัดงานกิจกรรม (Event Manager) จะต้องจัดเตรียมสถานที่ สื่อการนำเสนอ และการจัดการเอกสารต่าง ๆ ที่เกี่ยวข้องกับกิจกรรม เป็นภาระต่อผู้จัดงานกิจกรรม นอกจากนี้ยังไม่สามารถจัดกิจกกรรมในรูปแบบ Hybrid ได้ ในกรณีที่ผู้เข้าร่วมงาน (Guest) ต้องการที่จะเยี่ยมชมผลงาน โครงการต่าง ๆ แบบออนไลน์ เพื่อให้ Virtual Money และแสดงความคิดเห็นต่าง ๆ และส่วนสุดท้ายการแสดงผลหลังสิ้นสุดกิจกรรมไม่สามารถแสดงได้ทันที ทำให้การประกาศผลของผลงานที่ได้รับการประเมินสูงสุดทำได้ล่าช้า

        จากเหตุผลข้างต้นที่กล่าวมานั้น คณะผู้จัดทำจึงได้จัดทำโครงงานเพื่อพัฒนาเว็บแอปพลิเคชันสำหรับการจัดกิจกรรม Gallery Walk ที่สามารถจัดกิจกรรมทั้งในรูปแบบ online และ hybrid ซึ่งสามารถให้ผู้จัดงานกิจกรรมสามารถจัดการกิจกรรมได้ง่ายขึ้น และสามารถให้ผู้เข้าร่วมงานสามารถเยี่ยมชมผลงาน โครงการต่าง ๆ ทั้ง online และ hybrid ซึ่งยังสามารถแสดงความคิดเห็นต่าง ๆ และส่วนสุดท้ายสามารถแสดงผลได้ทันทีว่าผลงานใดได้รับการประเมินสูงสุดหลังเสร็จสิ้นงานกิจกรรม
    \end{abstractTH}

    \begin{abstract}

        Gallery Walk is the traditional way of learning and viewing works. This is a learning technique that encourages students to exchange information. Shared knowledge or experience by using the work that each group of learners jointly created as a medium for transferring and using a discussion board as a medium for receiving feedback and suggestions from others. To use that feedback to develop the group's work or complete the learners' own knowledge and understanding.

        At present, organizing an activity in the form of a gallery walk is difficult to manage. This is because the event organizer (the event manager) must prepare the venue. communication media and managing various documents related to activities. It is a burden on event organizers. In addition, activities cannot be organized in a hybrid format in cases where attendees (guests) would like to visit the various projects online in order to provide virtual money and express various opinions. And the last part is that it cannot immediately show which work received the highest evaluation after the event has been completed.

        From the above-mentioned The organizing team therefore created a project to develop a web application for organizing Gallery Walk activities in a hybrid format, which can allow event organizers to manage their activities more easily. And it can allow attendees to visit the various projects online and express various opinions, and the last part can immediately show which work received the highest evaluation after the event has finished.
    \end{abstract}

    \iffalse
        \begin{dedication}
            This document is dedicated to all Chiang Mai University students.

            Dedication page is optional.
        \end{dedication}
    \fi % \iffalse

    \begin{acknowledgments}
        
        
        โครงงานนี้ได้เสร็จสมบูรณ์ด้วยความกรุณาอย่างยิ่งจาก ผศ.โดม โพธิกานนท์ ผู้ที่ได้มอบเวลาและความกรุณาเพื่อเป็นที่ปรึกษาในโครงงานนี้ ได้ให้คำแนะนำที่มีคุณค่าและแนวทางที่ช่วยให้โครงงานเกิดขึ้นอย่างสมบูรณ์และประสบความสำเร็จ
        
        ไม่เพียงแต่นั้นเท่านั้น ขอขอบคุณอย่างสูงสำหรับคำปรึกษาจาก อ.ดร.ชินวัตร อิศราดิสัยกุล และ ผศ.ดร.ธนาทิพย์ จันทร์คง ที่ได้มอบคำแนะนำและเสนอแนวทางที่มีประสิทธิภาพ ทำให้โครงงานเกิดประสิทธิภาพมากขึ้น
        
        นอกจากนี้ เราต้องขอบคุณเพื่อนร่วมทีมทุกคนที่ให้การสนับสนุนและกำลังใจ และขอบคุณครอบครัวที่เป็นกำลังใจสำคัญในการผ่านช่วงเวลาที่ท้าทายไปด้วยกัน ขอบคุณทุกท่านที่เคยให้คำแนะนำและความช่วยเหลือ โดยที่ไม่ระบุนาม ทุกคำปรึกษาและการสนับสนุนที่มีค่ามากสำหรับเรา
        
        หากหนังสือโครงงานเล่มนี้มีข้อผิดพลาดประการใดผู้จัดทําขอน้อมรับด้วย ความยินดียิ่งและขออภัยมา ณ ที่นี้ด้วย
        \acksign{2024}{3}{20}
    \end{acknowledgments}%
\fi % \ifproject

\contentspage

\ifproject
    \figurelistpage

    \tablelistpage
\fi % \ifproject

% \abbrlist % this page is optional

% \symlist % this page is optional

% \preface % this section is optional


\pagestyle{empty}\cleardoublepage
\normalspacing \setcounter{page}{1} \pagenumbering{arabic} \pagestyle{cpecmu}

\chapter{\ifenglish Introduction\else บทนำ\fi}

\section{\ifenglish Project rationale\else ที่มาของโครงงาน\fi}

โครงการ Gallery Walk มีความหมายในการให้ผู้เรียนมีโอกาสแลกเปลี่ยนความรู้และประสบการณ์ผ่านผลงานที่สร้างขึ้นเป็นสื่อกลาง และการใช้กระดานสนทนาเพื่อรับคำแนะนำและคำชมเชยจากผู้อื่น เพื่อให้สามารถพัฒนาผลงานหรือเพิ่มเติมความรู้แก่ตนเองได้

การจัดกิจกรรมแบบ Gallery Walk ในปัจจุบันมีความยุ่งยากในการจัดการ เนื่องจากต้องมีการจัดเตรียมสถานที่ สื่อการสื่อสาร และการจัดการเอกสารต่าง ๆ ซึ่งเป็นภาระต่อผู้จัดงาน นอกจากนี้ยังไม่สามารถจัดกิจกรรมแบบ Hybrid ได้ โดยเฉพาะในกรณีที่ผู้เข้าร่วมต้องการที่จะเยี่ยมชมผลงานแบบออนไลน์และให้คำแนะนำ

ดังนั้น เพื่อแก้ไขปัญหาดังกล่าว ทีมผู้จัดงานได้พัฒนาเว็บแอปพลิเคชันสำหรับการจัดกิจกรรม Gallery Walk ในรูปแบบ Hybrid เพื่อให้การจัดการกิจกรรมเป็นไปอย่างมีประสิทธิภาพ และเพื่อให้ผู้เข้าร่วมสามารถเยี่ยมชมผลงานและให้ความเห็นออนไลน์ได้ นอกจากนี้ยังสามารถแสดงผลการประเมินได้ทันทีหลังจากเสร็จสิ้นกิจกรรม

\section{\ifenglish Objectives\else วัตถุประสงค์ของโครงงาน\fi}
\begin{enumerate}
    \item เพื่อให้การจัดกิจกรรมของผู้จัดกิจกรรมสะดวกและมีประสิทธิภาพมากขึ้น
    \item เพื่อสร้างช่องทางสําหรับการดําเนินกิจกรรมทางออนไลน์ให้กับผู้เข้าร่วมกิจกรรม สามารถจัดกิจกรรมในรูปแบบ Hybrid ได้
    \item การประเมินที่เน้นการมีส่วนร่วมและรับความคิดเห็น ซึ่งส่งเสริมการพัฒนาผลงานของผู้นำเสนอโครงการ
\end{enumerate}

\section{\ifenglish Project scope\else ขอบเขตของโครงงาน\fi}

\subsection{\ifenglish Hardware scope\else ขอบเขตด้านฮาร์ดแวร์\fi}
โครงการนี้ต้องการฮาร์ดแวร์ต่อไปนี้ จึงจะสามารถใช้งานได้อย่างมีประสิทธิภาพ
\begin{itemize}
    \item คอมพิวเตอร์ส่วนบุคคลหรือโทรศัพท์มือถือที่สามารถใช้งานเว็บเบราว์เซอร์ได้
\end{itemize}

\subsection{\ifenglish Software scope\else ขอบเขตด้านซอฟต์แวร์\fi}
โครงการนี้ต้องการซอฟต์แวร์ต่อไปนี้ จึงจะสามารถใช้งานได้อย่างมีประสิทธิภาพ
\begin{itemize}
    \item สามารถใช้งานเว็บไซต์บนระบบปฏิบัติการทั่วไปได้ เช่น Windows, macOS, Linux, Android, iOS และอื่น ๆ
\end{itemize}

\section{\ifenglish Expected outcomes\else ประโยชน์ที่ได้รับ\fi}
ผู้ใช้งาน
\begin{itemize}
    \item สามารถใช้งานเว็บแอปพลิเคชันเพื่อตรวจคัดกรองและเฝ้าระวังการเกิดรอยโรคก่อนมะเร็งและมะเร็งช่องปากได้
    \item สามารถเข้าถึงการรักษาทางการแพทย์ได้อย่างรวดเร็ว หลังจากที่ผู้ใช้งานได้รับการตรวจคัดกรองและเฝ้าระวังการเกิดรอยโรคก่อนมะเร็งและมะเร็งช่องปากโดยเว็บแอปพลิเคชัน
\end{itemize}
ผู้พัฒนา
\begin{itemize}
    \item ได้รับความรู้และความเข้าใจในการพัฒนาเว็บแอปพลิเคชันเพื่อรองรับระบบปัญญาประดิษฐ์ (AI)
    \item ได้ฝึกทักษะในการพัฒนาเว็บแอปพลิเคชันเพื่อรองรับระบบปัญญาประดิษฐ์ (AI)
    \item ได้ฝึกทักษะในการทำงานเป็นทีมและทักษะในการวิเคราะห์และแก้ไขปัญหาที่อาจเกิดขึ้นในการพัฒนา
\end{itemize}

\section{\ifenglish Technology and tools\else เทคโนโลยีและเครื่องมือที่ใช้\fi}

\subsection{\ifenglish Hardware technology\else เทคโนโลยีด้านฮาร์ดแวร์\fi}

\subsection{\ifenglish Software technology\else เทคโนโลยีด้านซอฟต์แวร์\fi}
\begin{itemize}
    \item ภาษาโปรแกรมมิ่ง: JavaScript, Python, HTML, CSS
    \item ฐานข้อมูล: MySQL
    \item เครื่องมือและเทคโนโลยี: NextJS, Tailwind CSS, Git, GitHub, Google Cloud Platform
\end{itemize}

\section{\ifenglish Project plan\else แผนการดำเนินงาน\fi}

\begin{plan}{6}{2023}{2}{2024}
    \planitem{6}{2023}{1}{2024}{ศึกษาค้นคว้าเกี่ยวกับเทคโนโลยีที่เกี่ยวข้อง}
    \planitem{9}{2023}{10}{2023}{ออกแบบ UI/UX ของเว็บแอปพลิเคชัน}
    \planitem{11}{2023}{1}{2024}{พัฒนาเว็บแอปพลิเคชัน}
    \planitem{1}{2024}{2}{2024}{ทดสอบและปรับปรุงเว็บแอปพลิเคชัน}
\end{plan}

\section{\ifenglish Roles and responsibilities\else บทบาทและความรับผิดชอบ\fi}
มีหน้าที่และความรับผิดชอบ ดังนี้


นายญาณาธิป ภู่สว่าง รหัสนักศึกษา 630612097 รับผิดชอบในการศึกษาค้นคว้าเทคโนโลยีที่เกี่ยวข้อง, ออกแบบโครงสร้างของเว็บแอปพลิเคชันและพัฒนาเว็บแอปพลิเคชัน


นายปัณฑ์ธร กันทรัพย์ รหัสนักศึกษา 630612105 รับผิดชอบในการศึกษาค้นคว้าเทคโนโลยีที่เกี่ยวข้อง, ออกแบบโครงสร้างของเว็บแอปพลิเคชันและพัฒนาเว็บแอปพลิเคชัน



% \section{\ifenglish%
%       Impacts of this project on society, health, safety, legal, and cultural issues
%   \else%
%       ผลกระทบด้านสังคม สุขภาพ ความปลอดภัย กฎหมาย และวัฒนธรรม
%   \fi}

% แนวทางและโยชน์ในการประยุกต์ใช้งานโครงงานกับงานในด้านอื่นๆ รวมถึงผลกระทบในด้านสังคมและสิ่งแวดล้อมจากการใช้ความรู้ทางวิศวกรรมที่ได้

\chapter{\ifenglish Background Knowledge and Theory\else ทฤษฎีที่เกี่ยวข้อง\fi}

\section{ด้านโครงสร้างเว็บแอปพลิเคชัน}
ในส่วนนี้จะอธิบายถึงโครงสร้างของเว็บแอปพลิเคชันที่ใช้ในการพัฒนา

\subsection{MVC Architecture}
MVC \cite{web:codebee} เป็นตัวย่อของคำว่า Model View Controller ใช้เรียกรูปแบบการพัฒนาซอฟต์แวร์ที่มีโครงสร้างซึ่งแบ่งออกมาเป็น 3 ส่วนหลัก ตามตัวย่อของชื่อ รูปแบบการพัฒนาซอฟต์แวร์แบบ MVC ถูกนำไปใช้ในขั้นตอนการพัฒนาหลากหลายภาษา
เพราะ MVC เป็นเพียงหลักการออกแบบโปรแกรม (Design Pattern) รูปแบบหนึ่งเท่านั้น ซึ่งเป็นที่นิยมมาก
ในการนำมาพัฒนาแอพพลิเคชั่นซอฟต์แวร์แต่ละแพลตฟอร์ม และประยุกต์ใช้ในอีกหลาย ๆ ด้าน
\subsubsection{ส่วนของ Model (M)}
model คือส่วนของการเก็บรวบรวมข้อมูล ไม่ว่าข้อมูลนั้น ๆ จะถูกจัดเก็บในรูปแบบใดก็ตาม ในฐานข้อมูล
แบบเป็น Object Class หรือที่นิยมเรียกกันว่า VO ( Value Object ) หรือเก็บเป็นไฟล์ข้อมูลเลย
เมื่อข้อมูลถูกโหลดเข้ามาจากที่ต่าง ๆ และเข้ามายังส่วนของโมเดล ตัวโมเดลจะทำการจัดการตระเตรียมข้อมูลให้เป็นรูปแบบที่เหมาะสม เพื่อรอการร้องขอข้อมูลจากส่วนของ Controller
\subsubsection{ส่วนของ View (V)}
view คือส่วนของการแสดงผล หรือส่วนที่จะปฏิสัมพันธ์กับผู้ใช้งาน ( User Interface ) หน้าที่ของ view
ในการเขียนโปรแกรมแบบ MVC คือคอยรับคำสั่งจากส่วนของ Controller และ End User เริ่มแรกเลยตัววิว
อาจจะได้รับคำสั่งจาก Controller ให้แสดงผลหน้า Home และเมื่อผู้ใช้งานหน้าเว็บกดปุ่มสั่งซื้อ View จะส่งข้อมูลไปให้ Controller เพื่อประมวลผลและแสดงบางอย่างจาก Action นั้น
\subsubsection{ส่วนของ Controller (C)}
controller คือส่วนของการเริ่มทำงาน และรับคำสั่ง โดยที่คำสั่งนั้นจะเกิดขึ้นในส่วนการติดต่อกับผู้ใช้งานคือ view
เมื่อผู้ใช้งานทำการ Interactive กับ UI view จะเกิดเหตุการณ์หรือข้อมูลบางอย่างขึ้น ตัววิวจะส่งข้อมูลนั้น
มายัง controller ตัว controller จะทำการประมวลผลโดยบางคำสั่งอาจจะต้องไปติดต่อกับ model ก่อน
เพื่อทำการประมวลผลข้อมูลอย่างถูกต้องเรียบร้อยแล้วก็จะส่งไปยัง view เพื่อแสดงผลตามคำสั่งที่ end user ร้องขอมา
Controller จะทำหน้าที่เป็นตัวกลางระหว่าง Model และ View ให้ทำงานร่วมกันอย่างมีประสิทธิภาพและตรงกับ
ความต้องการของ End User มากที่สุด

\subsection{RESTful API} 
RESTful API \cite{web:RESTful} เป็นอินเทอร์เฟซที่ระบบคอมพิวเตอร์สองระบบใช้เพื่อแลกเปลี่ยนข้อมูลผ่านอินเทอร์เน็ตได้อย่างปลอดภัย แอปพลิเคชันทางธุรกิจส่วนใหญ่ต้องสื่อสารกับแอปพลิเคชันภายในอื่นๆ และของบุคคลที่สามเพื่อทำงานต่างๆ ตัวอย่างเช่น หากต้องการสร้างสลิปเงินเดือน ระบบบัญชีภายในของคุณต้องแบ่งปันข้อมูลกับระบบธนาคารของลูกค้าเพื่อออกใบแจ้งหนี้และสื่อสารกับแอปพลิเคชันบันทึกเวลาปฏิบัติงานภายในโดยอัตโนมัติ RESTful API ให้การสนับสนุนการแลกเปลี่ยนข้อมูลนี้เพราะเป็นระบบที่มีมาตรฐานการสื่อสารระหว่างซอฟต์แวร์ที่ปลอดภัย เสถียร และมีประสิทธิภาพ
\subsubsection{API (Application Programming Interface)}
ส่วนต่อประสานโปรแกรมประยุกต์ (Application Programming Interface หรือ API) กำหนดกฎที่คุณต้องปฏิบัติตามเพื่อสื่อสารกับระบบซอฟต์แวร์อื่น โดยนักพัฒนาเปิดเผยหรือสร้าง API เพื่อให้แอปพลิเคชันอื่นสามารถสื่อสารกับแอปพลิเคชันของตนได้ทางโปรแกรม ตัวอย่างเช่น แอปพลิเคชันบันทึกเวลาปฏิบัติงานแสดง API ที่ขอชื่อเต็มของพนักงานและช่วงวันที่ เมื่อได้รับข้อมูลนี้แล้ว ระบบจะประมวลผลบันทึกเวลาปฏิบัติงานของพนักงานเป็นการภายใน และส่งกลับจำนวนชั่วโมงที่ทำงานในช่วงวันที่ดังกล่าว
ทั้งนี้คุณสามารถมองได้ว่า API เว็บเป็นเกตเวย์ระหว่างไคลเอ็นต์และทรัพยากรบนเว็บ

ไคลเอ็นต์
ไคลเอ็นต์คือผู้ใช้ที่ต้องการเข้าถึงข้อมูลจากเว็บ โดยไคลเอ็นต์อาจเป็นบุคคลหรือระบบซอฟต์แวร์ที่ใช้ API ก็ได้ ตัวอย่างเช่น นักพัฒนาสามารถเขียนโปรแกรมที่เข้าถึงข้อมูลสภาพอากาศจากระบบสภาพอากาศ หรือคุณสามารถเข้าถึงข้อมูลเดียวกันจากเบราว์เซอร์เมื่อคุณเยี่ยมชมเว็บไซต์รายงานสภาพอากาศได้โดยตรง

ทรัพยากร
ทรัพยากรคือข้อมูลที่แอปพลิเคชันต่างๆ มอบให้แก่ไคลเอ็นต์ โดยทรัพยากรอาจเป็นรูปภาพ วิดีโอ ข้อความ ตัวเลข หรือข้อมูลประเภทใดก็ได้ ทั้งนี้เครื่องคอมพิวเตอร์ที่มอบทรัพยากรให้แก่ไคลเอ็นต์นั้นเรียกอีกอย่างว่าเซิร์ฟเวอร์ องค์กรต่างๆ ใช้ API เพื่อแบ่งปันทรัพยากรและให้บริการเว็บในขณะที่ยังคงดูแลรักษาความปลอดภัย การควบคุม และการรับรองความถูกต้องไปพร้อมกัน นอกจากนี้ API ยังช่วยให้ลูกค้าระบุได้ว่าไคลเอ็นต์ใดสามารถเข้าถึงทรัพยากรภายในที่เฉพาะเจาะจงได้
\subsubsection{REST (Representational State Transfer)}
REST เป็นสถาปัตยกรรมซอฟต์แวร์ที่กำหนดเงื่อนไขว่า API ควรทำงานอย่างไร โดยแต่แรกเริ่มนั้น มีการสร้าง REST ขึ้นเพื่อเป็นแนวทางในการจัดการการสื่อสารบนเครือข่ายที่ซับซ้อน เช่น อินเทอร์เน็ต คุณสามารถใช้สถาปัตยกรรม REST เพื่อรองรับการสื่อสารที่มีประสิทธิภาพสูงและเชื่อถือได้ในทุกระดับ คุณยังสามารถใช้และปรับเปลี่ยนสถาปัตยกรรมได้อย่างง่ายดาย โดยนำความสามารถในการมองเห็นและการเคลื่อนย้ายข้ามแพลตฟอร์มมาสู่ทุกระบบ API

นักพัฒนา API สามารถออกแบบ API ได้โดยใช้สถาปัตยกรรมต่างๆ โดย API ที่เป็นไปตามรูปแบบสถาปัตยกรรม REST เรียกว่า REST API บริการเว็บที่ใช้สถาปัตยกรรม REST เรียกว่าบริการเว็บ RESTful คำว่า RESTful API โดยทั่วไปหมายถึง API เว็บแบบ RESTful อย่างไรก็ตาม คุณสามารถใช้คำว่า REST API และ RESTful API แทนกันได้
\subsection{ระบบฐานข้อมูล (Database System)}
ระบบฐานข้อมูล (Database System) \cite{web:database} คือ ระบบที่รวบรวมข้อมูลต่าง ๆ ที่เกี่ยวข้องกันเข้าไว้ด้วยกันอย่างมีระบบ มีความสัมพันธ์ระหว่างข้อมูลต่าง ๆ ที่ชัดเจน ในระบบฐานข้อมูลจะประกอบด้วยแฟ้มข้อมูลหลายแฟ้มที่มีข้อมูลเกี่ยวข้องสัมพันธ์กันเข้าไว้ด้วยกันอย่างเป็นระบบและเปิดโอกาสให้ผู้ใช้สามารถใช้งาน
และดูแลรักษาป้องกันข้อมูลเหล่านี้ได้อย่างมีประสิทธิภาพ โดยมีซอฟต์แวร์ที่เปรียบเสมือนสื่อกลางระหว่าง
ผู้ใช้และโปรแกรมต่าง ๆ ที่เกี่ยวข้องกับการใช้ฐานข้อมูล เรียกว่า ระบบจัดการฐานข้อมูล หรือ DBMS (data base management system)มีหน้าที่ช่วยให้ผู้ใช้เข้าถึงข้อมูลได้ง่ายสะดวกและมีประสิทธิภาพ การเข้าถึงข้อมูลของผู้ใช้อาจเป็นการสร้างฐานข้อมูล การแก้ไขฐานข้อมูล หรือการตั้งคำถามเพื่อให้ได้ข้อมูลมา โดยผู้ใช้ไม่จำเป็นต้องรับรู้เกี่ยวกับรายละเอียดภายในโครงสร้างของฐานข้อมูล
\subsubsection{ประโยชน์ของฐานข้อมูล}
\begin{enumerate}
    \item ลดการเก็บข้อมูลที่ซ้ำซ้อน 

    ข้อมูลบางชุดที่อยู่ในรูปของแฟ้มข้อมูลอาจมีปรากฏอยู่หลาย ๆ แห่ง เพราะมีผู้ใช้ข้อมูลชุดนี้หลายคน เมื่อใช้ระบบฐานข้อมูลแล้วจะช่วยให้ความซ้ำซ้อนของข้อมูลลดน้อยลง
\item รักษาความถูกต้องของข้อมูล 

เนื่องจากฐานข้อมูลมีเพียงฐานข้อมูลเดียว ใน
กรณีที่มีข้อมูลชุดเดียวกันปรากฏอยู่หลายแห่งในฐานข้อมูล ข้อมูลเหล่านี้จะต้องตรงกัน ถ้ามีการแก้ไขข้อมูลนี้ทุก ๆ แห่งที่ข้อมูลปรากฏอยู่จะแก้ไขให้ถูกต้องตามกันหมดโดยอัตโนมัติด้วยระบบจัดการฐานข้อมูล
\item การป้องกันและรักษาความปลอดภัยให้กับข้อมูลทำได้อย่างสะดวก 

การป้องกันและรักษาความปลอดภัยกับข้อมูลระบบฐานข้อมูลจะให้เฉพาะผู้ที่เกี่ยวข้องเท่านั้น
ซึ่งก่อให้เกิดความปลอดภัย (security) ของข้อมูลด้วย
\end{enumerate}
\section{Technology}
ในส่วนนี้จะอธิบายถึงเทคโนโลยีที่ใช้ในการพัฒนาเว็บแอปพลิเคชัน

\subsection{HTML}
HTML \cite{web:html} ย่อมาจาก Hyper Text Markup Language คือภาษาคอมพิวเตอร์ที่ใช้ในการแสดงผลของเอกสารบน website หรือที่เราเรียกกันว่าเว็บเพจ ถูกพัฒนาและกำหนดมาตรฐานโดยองค์กร World Wide Web Consortium (W3C) และจากการพัฒนาทางด้าน Software ของ Microsoft ทำให้ภาษา HTML เป็นอีกภาษาหนึ่งที่ใช้เขียนโปรแกรมได้ หรือที่เรียกว่า HTML Application  
HTML เป็นภาษาประเภท Markup สำหรับการการสร้างเว็บเพจ โดยใช้ภาษา HTML สามารถทำโดยใช้โปรแกรม Text Editor ต่างๆ เช่น VS Code, Vim หรือจะอาศัยโปรแกรมที่เป็นเครื่องมือช่วยสร้างเว็บเพจ เช่น Dream Weaver ซึ่งอํานวยความสะดวกในการสร้างหน้า HTML ส่วนการเรียกใช้งานหรือทดสอบการทำงานของเอกสาร HTML จะใช้โปรแกรม web browser เช่น Google Chrome, Microsoft Edge, Mozilla Firefox, Safari และ Opera เป็นต้น


\subsection{CSS}

\subsection{TypeScript}

\subsection{Tailwind CSS}

\subsection{Next.js}

\subsection{MySQL}

\subsection{Docker}

\subsection{JSON}

\subsection{Relational Database Management System (RDBMS)}



\section{ด้าน User Interface}
ในส่วนนี้จะอธิบายถึงการออกแบบ User Interface ของเว็บแอปพลิเคชัน


% \noindent
% where $\omega$ is the frequency of the plasmon, $c$ is the speed of
% light, $\varepsilon_m$ is the dielectric constant of the metal,
% $\varepsilon_i$ is the dielectric constant of neighboring insulator,
% and $\varepsilon_\mathit{air}$ is the dielectric constant of air.

% \section{About using figures in your report}

% % define a command that produces some filler text, the lorem ipsum.
% \newcommand{\loremipsum}{
%   \textit{Lorem ipsum dolor sit amet, consectetur adipisicing elit, sed do
%   eiusmod tempor incididunt ut labore et dolore magna aliqua. Ut enim ad
%   minim veniam, quis nostrud exercitation ullamco laboris nisi ut
%   aliquip ex ea commodo consequat. Duis aute irure dolor in
%   reprehenderit in voluptate velit esse cillum dolore eu fugiat nulla
%   pariatur. Excepteur sint occaecat cupidatat non proident, sunt in
%   culpa qui officia deserunt mollit anim id est laborum.}\par}

% \begin{figure}
%   \centering

%   \fbox{
%      \parbox{.6\textwidth}{\loremipsum}
%   }

%   % To include an image in the figure, say myimage.pdf, you could use
%   % the following code. Look up the documentation for the package
%   % graphicx for more information.
%   % \includegraphics[width=\textwidth]{myimage}

%   \caption[Sample figure]{This figure is a sample containing \gls{lorem ipsum},
%   showing you how you can include figures and glossary in your report.
%   You can specify a shorter caption that will appear in the List of Figures.}
%   \label{fig:sample-figure}
% \end{figure}

% Using \verb.\label. and \verb.\ref. commands allows us to refer to
% figures easily. If we can refer to Figures
% \ref{fig:walrus} and \ref{fig:sample-figure} by name in the {\LaTeX}
% source code, then we will not need to update the code that refers to it
% even if the placement or ordering of the figures changes.

% \loremipsum\loremipsum

% % This code demonstrates how to get a landscape table or figure. It
% % uses the package lscape to turn everything but the page number into
% % landscape orientation. Everything should be included within an
% % \afterpage{ .... } to avoid causing a page break too early.
% \afterpage{
%   \begin{landscape}
%   \begin{table}
%     \caption{Sample landscape table}
%     \label{tab:sample-table}

%     \centering

%     \begin{tabular}{c||c|c}
%         Year & A & B \\
%         \hline\hline
%         1989 & 12 & 23 \\
%         1990 & 4 & 9 \\
%         1991 & 3 & 6 \\
%     \end{tabular}
%   \end{table}
%   \end{landscape}
% }

% \loremipsum\loremipsum\loremipsum

% \section{Overfull hbox}

% When the \verb.semifinal. option is passed to the \verb.cpecmu. document class,
% any line that is longer than the line width, i.e., an overfull hbox, will be
% highlighted with a black solid rule:
% \begin{center}
% \begin{minipage}{2em}
% juxtaposition
% \end{minipage}
% \end{center}

% \section{\ifenglish%
% \ifcpe CPE \else ISNE \fi knowledge used, applied, or integrated in this project
% \else%
% ความรู้ตามหลักสูตรซึ่งถูกนำมาใช้หรือบูรณาการในโครงงาน
% \fi
% }

% อธิบายถึงความรู้ และแนวทางการนำความรู้ต่างๆ ที่ได้เรียนตามหลักสูตร ซึ่งถูกนำมาใช้ในโครงงาน

% \section{\ifenglish%
% Extracurricular knowledge used, applied, or integrated in this project
% \else%
% ความรู้นอกหลักสูตรซึ่งถูกนำมาใช้หรือบูรณาการในโครงงาน
% \fi
% }

% อธิบายถึงความรู้ต่างๆ ที่เรียนรู้ด้วยตนเอง และแนวทางการนำความรู้เหล่านั้นมาใช้ในโครงงาน

\chapter{\ifproject%
      \ifenglish Project Structure and Methodology\else โครงสร้างและขั้นตอนการทำงาน\fi
  \else%
      \ifenglish Project Structure\else โครงสร้างของโครงงาน\fi
  \fi
 }

ในบทนี้จะกล่าวถึงหลักการ และการออกแบบระบบ

\makeatletter

% \renewcommand\section{\@startsection {section}{1}{\z@}%
%                                    {13.5ex \@plus -1ex \@minus -.2ex}%
%                                    {2.3ex \@plus.2ex}%
%                                    {\normalfont\large\bfseries}}

\makeatother
%\vspace{2ex}
% \titleformat{\section}{\normalfont\bfseries}{\thesection}{1em}{}
% \titlespacing*{\section}{0pt}{10ex}{0pt}

\section{หลักการทำงานของระบบ}


% \begin{figure}
% \begin{center}
% \includegraphics{800px-Briny_Beach.jpg}
% \end{center}
% \caption[Poem]{The Walrus and the Carpenter}
% \label{fig:walrus}
% \end{figure}
\subsection{การทำงานของระบบ (System Architecture)}
\begin{figure}[h]
    \begin{center}
        \includegraphics[width=0.9\textwidth]{img/system achitecture.png}
    \end{center}
    \caption{System Architecture}
    \label{fig:system_overview}
\end{figure}

จากรูปที่ \ref{fig:system_overview} จะเป็นภาพรวมของระบบที่เราได้ทำการออกแบบขึ้น โดยมีรายละเอียดดังนี้
\begin{itemize}
    \item Frontend

          ส่วนหน้าบ้าน (Frontend) เป็นส่วนการพัฒนาเพื่อแสดง User Interface (UI) โดยโครงการนี้ได้ใช้เทคโนโลยี React ร่วมกับ Mantine UI ในการออกแบบและสร้างส่วนติดต่อผู้ใช้ (UI) โดยมีหน้าที่แสดงผลลัพธ์ต่อผู้ใช้ในรูปแบบที่เข้าใจง่าย รับข้อมูลป้อนจากผู้ใช้ผ่านอินเทอร์เฟซต่าง ๆ เช่น ปุ่ม ฟิลด์ข้อความ ฯลฯ และสื่อสารกับ API เพื่อส่งคำร้องขอและรับผลลัพธ์
    \item Backend

          ส่วนหลังบ้าน (Backend) เป็นส่วนที่ทำหน้าที่รับข้อมูลจากผู้ใช้ จากนั้นทำการประมวลผลข้อมูล และส่งข้อมูลกลับไปยังผู้ใช้ โดยโครงการนี้ได้ใช้เทคโนโลยี Fastify เว็บเฟรมเวิร์ค Node.js ร่วมกับ Prisma ORM ในการจัดการข้อมูลในฐานข้อมูล รวมถึงทำการสื่อสารกับฐานข้อมูล PostgreSQL และ Minio (Object Storage) ในการจัดการข้อมูลที่เป็นไฟล์
\end{itemize}

\subsection{เส้นทางของผู้ใช้ (User Flow)}
\begin{figure}[h]
    \begin{center}
        \includegraphics[width=0.9\textwidth]{img/userflow.png}
    \end{center}
    \caption{User Flow}
    \label{fig:user_flow}
\end{figure}

จากรูปที่ \ref{fig:user_flow} จะเป็นเส้นทางของผู้ใช้งานที่เข้าใช้งานระบบ โดยมีรายละเอียดดังนี้

\subsubsection{ผู้สร้างงานกิจกรรม (Event Manager)}
\begin{itemize}
    \item ผู้ใช้เข้าสู่ระบบโดยใช้ Google Account
    \item ผู้ใช้สร้าง Event ใหม่ โดยป้อนข้อมูลต่างๆ เช่น ชื่องานกิจกรรม รายละเอียด วันเวลา สถานที่ เป็นต้น
    \item ผู้ใช้สร้าง Event เสร็จสิ้น จะมี QR Code หรือ Access Link สำหรับนำไปให้ผู้นำเสนอโครงการ (Presenter) สำหรับเพิ่มโครงการของตนและผู้เข้าร่วมกิจกรรม (Guest) สำหรับเข้าร่วมกิจกรรม
    \item ผู้ใช้สามารถดูผลลัพธ์ (Chart และ Ranking) ของโครงการต่าง ๆ ที่เข้าร่วมกิจกรรมในระหว่างจัดงานกิจกรรม

\end{itemize}

\subsubsection{ผู้นำเสนอโครงการ (Presenter)}
\begin{itemize}
    \item ผู้ใช้เข้าสู่ระบบโดยใช้ Google Account
    \item ผู้ใช้สร้าง Project ใหม่ โดยป้อนข้อมูลต่าง ๆ เช่น ชื่อโครงการ รายละเอียด รูปภาพ ลิงก์ และไฟล์อื่น ๆ ที่เกี่ยวข้อง
    \item ผู้ใช้ดูผลลัพธ์ว่าโครงการของตนได้รับ Virtual Money จากผู้เข้าร่วมกิจกรรม (Guest) และความคิดเห็นเกี่ยวกับโครงการของตนหลังจากเสร็จสิ้นงานกิจกรรม

\end{itemize}

\subsubsection{ผู้เข้าร่วมกิจกรรม (Guest)}
\begin{itemize}
    \item ผู้ใช้เข้าร่วม Event โดยใช้ QR COde หรือ Access Link ที่ได้รับจาก Event Manager
    \item ผู้ใช้เข้าสู่ระบบโดยใช้ Google Account
    \item ผู้ใช้ดูรายละเอียดของงานกิจกรรมที่จัดขึ้น รวมถึงโครงการที่เข้าร่วมกิจกรรม
    \item ผู้ใช้สามารถให้ Virtual Money และแสดงความคิดเห็นเกี่ยวกับ Project ต่าง ๆ ที่เข้าร่วมกิจกรรมได้
\end{itemize}

\newpage
\subsection{โครงสร้างฐานข้อมูล (Database Schema)}
\begin{figure}[hb]
    \begin{center}
        \includegraphics[width=0.8\textwidth]{img/database schama.png}
    \end{center}
    \caption{Database Schema}
    \label{fig:data_schema}
\end{figure}
จากรูปที่ \ref{fig:data_schema} จะเป็นโครงสร้างของฐานข้อมูลที่ใช้ในระบบ โดยมีทั้งหมด 9  ตาราง ดังนี้


\begin{table}[hb]
    \centering
    \begin{tabular}{|c|c|c|}
        \hline
        ชื่อคอลัมน์               & ชนิดข้อมูล   & คำอธิบาย          \\ \hline
        \verb |first_name_th| & VARCHAR   & ชื่อภาษาไทย       \\ \hline
        \verb |last_name_th|  & VARCHAR   & นามสกุลภาษาไทย   \\ \hline
        \verb |first_name_en| & VARCHAR   & ชื่อภาษาอังกฤษ     \\ \hline
        \verb |last_name_en|  & VARCHAR   & นามสกุลภาษาอังกฤษ \\ \hline
        \verb |email|         & VARCHAR   & อีเมล์            \\ \hline
        \verb |affiliation|   & TEXT      & สังกัด            \\ \hline
        \verb |profile_pic|   & TEXT      & รูปภาพโปรไฟล์     \\ \hline
        \verb |role|          & VARCHAR   & บทบาทของผู้ใช้งาน  \\ \hline
        \verb |created_at|    & TIMESTAMP & วันที่สร้างข้อมูล     \\ \hline
        \verb |updated_at|    & TIMESTAMP & วันที่อัปเดตข้อมูล    \\ \hline
    \end{tabular}
    \caption{ตารางข้อมูลผู้ใช้งาน (Users)}
    \label{tab:user_data}
\end{table}

\begin{table}[ht]
    \centering
    \begin{tabular}{|c|c|c|}
        \hline
        ชื่อคอลัมน์                  & ชนิดข้อมูล   & คำอธิบาย                                 \\ \hline
        \verb |event_name|       & VARCHAR   & ชื่องานกิจกรรม                            \\ \hline
        \verb |start_date|       & DATE      & วันที่เริ่มงานกิจกรรม                        \\ \hline
        \verb |end_date|         & DATE      & วันที่สิ้นสุดงานกิจกรรม                       \\ \hline
        \verb |description|      & TEXT      & รายละเอียดงานกิจกรรม                     \\ \hline
        \verb |submit_start|     & DATE      & วันที่เริ่มรับส่งโครงการ                      \\ \hline
        \verb |submit_end|       & DATE      & วันที่สิ้นสุดรับส่งโครงการ                     \\ \hline
        \verb |number_of_member| & INTEGER   & จำนวนสมาชิกของโครงการ                    \\ \hline
        \verb |virtual_money|    & INTEGER   & จำนวนเงินเสมือนที่จะให้แขกผู้เข้าร่วมงาน (Guest) \\ \hline
        \verb |unit_money|       & VARCHAR   & หน่วยเงินเสมือนที่จะให้แขกผู้เข้าร่วมงาน (Guest) \\ \hline
        \verb |published|        & BOOLEAN   & สถานะการเผยแพร่ของงานกิจกรรม             \\ \hline
        \verb |organization|     & TEXT      & สังกัดของงานกิจกรรม                       \\ \hline
        \verb |video_link|       & TEXT      & ลิงก์วิดีโอที่เกี่ยวข้องกับงานกิจกรรม             \\ \hline
        \verb |user_id|          & INTEGER   & รหัสผู้ใช้งานที่สร้างงานกิจกรรม                \\ \hline
        \verb |location|         & TEXT      & สถานที่จัดงานกิจกรรม                       \\ \hline
        \verb |created_at|       & TIMESTAMP & วันที่สร้างข้อมูล                            \\ \hline
        \verb |updated_at|       & TIMESTAMP & วันที่อัปเดตข้อมูล                           \\ \hline
    \end{tabular}
    \caption{ตารางข้อมูลงานกิจกรรม (Events)}
    \label{tab:event_data}
\end{table}

\begin{table}[hb]
    \centering
    \begin{tabular}{|c|c|c|}
        \hline
        ชื่อคอลัมน์             & ชนิดข้อมูล   & คำอธิบาย                     \\ \hline
        \verb |title|       & VARCHAR   & ชื่อโครงการ                  \\ \hline
        \verb |description| & TEXT      & รายละเอียดของโครงการ        \\ \hline
        \verb |user_id|     & INTEGER   & รหัสผู้ใช้งานที่สร้างโครงการ      \\ \hline
        \verb |event_id|    & INTEGER   & รหัสงานกิจกรรมที่โครงการเข้าร่วม \\ \hline
        \verb |created_at|  & TIMESTAMP & วันที่สร้างข้อมูล                \\ \hline
    \end{tabular}
    \caption{ตารางข้อมูลโครงการ (Projects)}
    \label{tab:project_data}
\end{table}

\begin{table}[ht]
    \centering
    \begin{tabular}{|c|c|c|}
        \hline
        ชื่อคอลัมน์            & ชนิดข้อมูล   & คำอธิบาย                   \\ \hline
        \verb |amount|     & INTEGER   & จำนวนเงินเสมือนที่ได้รับ        \\ \hline
        \verb |project_id| & INTEGER   & รหัสโครงการที่ได้รับเงิน       \\ \hline
        \verb |event_id|   & INTEGER   & รหัสงานกิจกรรมที่ได้รับเงิน     \\ \hline
        \verb |guest_id|   & INTEGER   & รหัสผู้เข้าร่วมกิจกรรมที่ได้รับเงิน \\ \hline
        \verb |created_at| & TIMESTAMP & วันที่สร้างข้อมูล              \\ \hline
        \verb |updated_at| & TIMESTAMP & วันที่อัปเดตข้อมูล             \\ \hline
    \end{tabular}
    \caption{ตารางข้อมูลเงินเสมือน (Virtual Money)}
    \label{tab:virtual_money_data}
\end{table}

\begin{table}[hb]
    \centering
    \begin{tabular}{|c|c|c|}
        \hline
        ชื่อคอลัมน์               & ชนิดข้อมูล   & คำอธิบาย           \\ \hline
        \verb |first_name_th| & VARCHAR   & ชื่อจริงภาษาไทย     \\ \hline
        \verb |last_name_th|  & VARCHAR   & นามสกุลภาษาไทย    \\ \hline
        \verb |first_name_en| & VARCHAR   & ชื่อจริงภาษาอังกฤษ   \\ \hline
        \verb |last_name_en|  & VARCHAR   & นามสกุลภาษาอังกฤษ  \\ \hline
        \verb |email|         & TEXT      & ที่อยู่อีเมล          \\ \hline
        \verb |profile_pic|   & TEXT      & รูปภาพโปรไฟล์      \\ \hline
        \verb |virtual_money| & INTEGER   & จำนวนเงินเสมือนที่มีอยู่ \\ \hline
        \verb |created_at|    & TIMESTAMP & วันที่สร้างข้อมูล      \\ \hline
    \end{tabular}
    \caption{ตารางข้อมูลแขกผู้เข้าร่วมกิจกรรม (Guests)}
    \label{tab:guest_data}
\end{table}

\begin{table}[ht]
    \centering
    \begin{tabular}{|c|c|c|}
        \hline
        ชื่อคอลัมน์               & ชนิดข้อมูล   & คำอธิบาย                \\ \hline
        \verb |event_id|      & INTEGER   & รหัสงานกิจกรรมที่ได้รับเงิน  \\ \hline
        \verb |thumbnail|     & TEXT      & รูปภาพตัวอย่างโครงการ    \\ \hline
        \verb |thumbnail_url| & TEXT      & ลิงก์รูปภาพตัวอย่างโครงการ \\ \hline
        \verb |created_at|    & TIMESTAMP & วันที่สร้างข้อมูล           \\ \hline
        \verb |updated_at|    & TIMESTAMP & วันที่อัปเดตข้อมูล          \\ \hline
    \end{tabular}
    \caption{ตารางข้อมูลภาพตัวอย่างโครงการ (Thumbnails)}
    \label{tab:thumbnail_data}
\end{table}

\begin{table}[hb]
    \centering
    \begin{tabular}{|c|c|c|}
        \hline
        ชื่อคอลัมน์               & ชนิดข้อมูล   & คำอธิบาย                     \\ \hline
        \verb |project_id|    & INTEGER   & รหัสโครงการที่เกี่ยวข้องกับเอกสาร \\ \hline
        \verb |document_name| & TEXT      & ชื่อเอกสาร                   \\ \hline
        \verb |document_url|  & TEXT      & ลิงก์เอกสาร                  \\ \hline
        \verb |created_at|    & TIMESTAMP & วันที่สร้างข้อมูล                \\ \hline
        \verb |updated_at|    & TIMESTAMP & วันที่อัปเดตข้อมูล               \\ \hline
    \end{tabular}
    \caption{ตารางข้อมูลเอกสารที่เกี่ยวข้องกับงานกิจกรรม (Documents)}
    \label{tab:document_data}
\end{table}

\begin{table}[ht]
    \centering
    \begin{tabular}{|c|c|c|}
        \hline
        ชื่อคอลัมน์                & ชนิดข้อมูล   & คำอธิบาย                        \\ \hline
        \verb |project_id|     & INTEGER   & รหัสโครงการที่เกี่ยวข้องกับความคิดเห็น \\ \hline
        \verb |comment_better| & TEXT      & ความคิดเห็นที่ดีของโครงการ         \\ \hline
        \verb |comment_idea|   & TEXT      & ความคิดเห็นที่เสนอไอเดียใหม่        \\ \hline
        \verb |comment_ilike|  & TEXT      & ความคิดเห็นที่ชอบของโครงการ       \\ \hline
        \verb |created_at|     & TIMESTAMP & วันที่สร้างข้อมูล                   \\ \hline
        \verb |updated_at|     & TIMESTAMP & วันที่อัปเดตข้อมูล                  \\ \hline
    \end{tabular}
    \caption{ตารางข้อมูลความคิดเห็นของผู้เข้าร่วมกิจกรรม (Comments)}
    \label{tab:comment_data}
\end{table}

\begin{table}[hb]
    \centering
    \begin{tabular}{|c|c|c|}
        \hline
        ชื่อคอลัมน์                   & ชนิดข้อมูล   & คำอธิบาย                    \\ \hline
        \verb |project_id|        & INTEGER   & รหัสโครงการที่เกี่ยวข้องกับรูปภาพ \\ \hline
        \verb |project_image|     & TEXT      & รูปภาพโครงการ              \\ \hline
        \verb |project_image_url| & TEXT      & ลิงก์รูปภาพโครงการ           \\ \hline
        \verb |created_at|        & TIMESTAMP & วันที่สร้างข้อมูล               \\ \hline
        \verb |updated_at|        & TIMESTAMP & วันที่อัปเดตข้อมูล              \\ \hline
    \end{tabular}
    \caption{ตารางข้อมูลรูปภาพที่เกี่ยวข้องกับโครงการ (Project Images)}
    \label{tab:project_images_data}
\end{table}

\clearpage % 
\section{ส่วนเชื่อมต่อระหว่างผู้ใช้งานกับระบบ (User Interface)}


โดยแบ่งเป็น 3 ผู้ใช้งาน คือ ผู้สร้างงานกิจกรรม (Event manager), ผู้นำเสนอผลงาน (Presenter), ผู้เข้าร่วมงาน (Guest)
\subsection{หน้าเริ่มต้นการใช้งาน (Homepage)}
โดยแบ่ง Tab เป็น Home, Service, About us และ Contact สำหรับ Event manager และ Presenter

\begin{figure}[h!] %
    \begin{center}
        \includegraphics[width=0.9\textwidth]{img/ui/homepage.png}
    \end{center}
    \caption{หน้า Homepage}
    \label{fig:homepage}

    \begin{center}
        \includegraphics[width=0.9\textwidth]{img/ui/service.png}
    \end{center}
    \caption{หน้า Service}
    \label{fig:service}
\end{figure}


\begin{figure}
    \begin{center}
        \includegraphics[width=0.9\textwidth]{img/ui/aboutus.png}
    \end{center}
    \caption{หน้า About us}
    \label{fig:aboutus}

    \begin{center}
        \includegraphics[width=0.9\textwidth]{img/ui/contact.png}
    \end{center}
    \caption{หน้า Contact}
    \label{fig:contact}
\end{figure}

\subsection{หน้าเข้าสู่ระบบ (Login)}
Login ด้วย Google account สำหรับ Event manager และ Presenter

\begin{figure}[h!] %
    \begin{center}
        \includegraphics[width=0.9\textwidth]{img/ui/login.png}
    \end{center}
    \caption{หน้า Login สำหรับ Event manager และ Presenter}
    \label{fig:login}
\end{figure}

\subsection{ผู้สร้างงานกิจกรรม (Event manager)}
\begin{figure}[h!] %
    \begin{center}
        \includegraphics[width=0.9\textwidth]{img/ui/event-dashboard.png}
    \end{center}
    \caption{หน้าแสดงกิจกรรมที่สร้าง (Event manager dashboard)}
    \label{fig:event-dashboard}
\end{figure}

\begin{figure}
    \begin{center}
        \includegraphics[width=0.9\textwidth]{img/ui/event-info.png}
    \end{center}
    \caption{หน้าแสดงข้อมูลรายละเอียดของกิจกรรมที่สร้าง โดยแบ่ง Tab เป็น Event information, Project และ Result}
    \label{fig:event-info}

    \begin{center}
        \includegraphics[width=0.9\textwidth]{img/ui/project-list.png}
    \end{center}
    \caption{หน้าแสดงรายการโครงการที่เข้าร่วมกิจกรรม}
    \label{fig:project-list}
\end{figure}

\begin{figure}[t]
    \begin{center}
        \includegraphics[width=0.9\textwidth]{img/ui/result.png}
    \end{center}
    \caption{หน้าแสดงผลลัพธ์ของโครงการที่เข้าร่วมกิจกรรม โดยแสดงเป็น Ranking และตาราง Virtual Money}
    \label{fig:result}

    \begin{center}
        \includegraphics[height=0.7\textwidth]{img/ui/ranking-table-result.png}
    \end{center}
    \caption{ตารางแสดงผลลัพธ์ของโครงการที่เข้าร่วมกิจกรรม}
    \label{fig:ranking-table-result}
\end{figure}

\clearpage
\subsubsection{สร้างกิจกรรม (Create Event)}
\begin{figure}[h]
    \begin{center}
        \includegraphics[width=0.9\textwidth]{img/ui/create-event-1.png}
    \end{center}
    \caption{หน้าสร้างกิจกรรม First step (Event information)}
    \label{fig:create-event}
    \begin{center}
        \includegraphics[width=0.9\textwidth]{img/ui/create-event-2.png}
    \end{center}
    \caption{หน้าสร้างกิจกรรม Second step (Images Dropzone)}
    \label{fig:create-event-2}
\end{figure}

\clearpage
\begin{figure}[ht]
    \begin{center}
        \includegraphics[width=0.9\textwidth]{img/ui/create-event-3.png}
    \end{center}
    \caption{หน้าสร้างกิจกรรม Final step (Check all information before submit)}
    \label{fig:create-event-3}
\end{figure}

\clearpage
\subsection{ผู้นำเสนอผลงาน (Presenter)}
หลังจากที่เข้าสู่งานกิจกรรมจาก QR Code หรือ Access Link ที่ได้รับจาก Event manager จะเข้าสู่หน้าเข้าสู่ระบบ (Login) และหลังจากที่เข้าสู่ระบบจะเข้าสู่หน้าแสดงข้อมูลรายละเอียดของงานกิจกรรมที่จัดขึ้น และผู้นำเสนอโครการสามารถเพิ่มโครงการเข้าร่วมงานกิจกรรมได้
\begin{figure}[h]
    \begin{center}
        \includegraphics[width=0.9\textwidth]{img/ui/add-project.png}
    \end{center}
    \caption{หน้าเพิ่มโครงการที่เข้าร่วมกิจกรรม (Add project)}
    \label{fig:add-project}

    \begin{center}
        \includegraphics[width=0.9\textwidth]{img/ui/presenter-dashboard.png}
    \end{center}
    \caption{หน้าแสดงโครงการที่สร้าง (Presenter dashboard)}
    \label{fig:presenter-dashboard}
\end{figure}

\clearpage
\begin{figure}[ht]
    \begin{center}
        \includegraphics[width=0.9\textwidth]{img/ui/project-info.png}
    \end{center}
    \caption{หน้าแสดงข้อมูลรายละเอียดของผลงานที่สร้างในกิจกรรม}
    \label{fig:project-info}
\end{figure}
\clearpage %

% \clearpage
\subsection{ผู้เข้าร่วมงาน (Guest)}
เมื่อเข้าสู่งานกิจกรรมจาก QR Code หรือ Access Link ที่ได้รับจาก Event manager จะเข้าสู่หน้าเข้าสู่ระบบ (Login) และหลังจากที่เข้าสู่ระบบจะเข้าสู่หน้าแสดงข้อมูลรายละเอียดของงานกิจกรรมที่จัดขึ้น และโครงการที่เข้าร่วมกิจกรรม
\begin{figure}[h]
    \begin{center}
        \includegraphics[width=0.9\textwidth]{img/ui/guest-login.png}
    \end{center}
    \caption{หน้าเข้าสู่ระบบ (Login) สำหรับผู้เข้าร่วมกิจกรรม (Guest)}
    \label{fig:guest-dashboard}

    \begin{center}
        \includegraphics[height=0.5\textwidth]{img/ui/guest-dashboard.png}
    \end{center}
    \caption{หน้าแสดงข้อมูลรายละเอียดของงานกิจกรรมที่จัดขึ้น และโครงการที่เข้าร่วมกิจกรรม}
    \label{fig:guest-dashboard}
\end{figure}

\begin{figure}
    \begin{center}
        \includegraphics[height=0.7\textwidth]{img/ui/give-virtual-money.png}
    \end{center}
    \caption{หน้าให้ Virtual Money และแสดงความคิดเห็นเกี่ยวกับโครงการที่เข้าร่วมกิจกรรม}
    \label{fig:guest-project-info}

    \begin{center}
        \includegraphics[height=0.7\textwidth]{img/ui/give-comment.png}
    \end{center}
    \caption{หน้าแสดงความคิดเห็นเกี่ยวกับโครงการที่เข้าร่วมกิจกรรม}
    \label{fig:give-comment}
\end{figure}
\chapter{\ifproject%
      \ifenglish Experimentation and Results\else การทดลองและผลลัพธ์\fi
  \else%
      \ifenglish System Evaluation\else การประเมินระบบ\fi
  \fi}

ในบทนี้จะทดสอบเกี่ยวกับการทำงานในฟังก์ชันหลัก ๆ

\section{การทดลองเกี่ยวกับการทำงานของระบบ}
การประเมินระบบจะประเมินโดยทดสอบกับกลุ่มผู้ใช้งานทั้ง 4 กลุ่ม ได้แก่ ผู้ใช้ทั่วไป, ทันตแพทย์, ทันตบุคลากร และอาสาสมัครสาธารณสุขประจําหมู่บ้าน (อสม.) โดยในการทดสอบระบบจะมีการประเมินผลการทดลองโดยใช้เกณฑ์ต่าง ๆ ดังนี้
\subsection{ผู้ใช้ทั่วไป}
ผู้ใช้ทั่วไปมักมีความต้องการใช้งานระบบที่เรียบง่าย ใช้งานง่าย ไม่ซับซ้อน ดังนั้น ในการทดสอบกับผู้ใช้ทั่วไป ควรเน้นการประเมินปัจจัยต่างๆ เช่น

\begin{itemize}
    \item ความน่าใช้งาน: Ease of use
    \item ความพึงพอใจของผู้ใช้งาน: User satisfaction
    \item ประโยชน์: Benefits


          ตัวอย่างวิธีการทดสอบกับผู้ใช้ทั่วไป ได้แก่
    \item ให้ผู้ใช้ทดสอบระบบและรวบรวมข้อมูลเกี่ยวกับประสบการณ์การใช้งาน เช่น ระยะเวลาในการดำเนินการแต่ละขั้นตอน ความสะดวกในการใช้งาน เป็นต้น
    \item ให้ผู้ใช้ตอบแบบสอบถามเกี่ยวกับความพึงพอใจต่อระบบ เช่น ความง่ายในการใช้งาน ความน่าสนใจของเนื้อหา เป็นต้น
    \item ให้ผู้ใช้ประเมินประโยชน์ที่ได้รับจากระบบ เช่น ช่วยให้ประหยัดเวลา ช่วยให้เข้าใจข้อมูลต่างๆ ได้ง่าย เป็นต้น
\end{itemize}


\subsection{ทันตแพทย์}

ทันตแพทย์มีความต้องการใช้งานระบบที่มีประสิทธิภาพ ถูกต้องแม่นยำและสามารถช่วยในตรวจคัดกรองมะเร็งช่องปากได้อย่างมีประสิทธิภาพ ดังนั้น ในการทดสอบกับทันตแพทย์ ควรเน้นการประเมินปัจจัยต่างๆ เช่น

\begin{itemize}
    \item ความน่าใช้งาน: Ease of use
    \item ความพึงพอใจของผู้ใช้งาน: User satisfaction
    \item ประโยชน์: Benefits
    \item การช่วยในการตรวจคัดกรองมะเร็งช่องปาก: Screening


          ตัวอย่างวิธีการทดสอบกับทันตแพทย์ ได้แก่

    \item ให้ทันตแพทย์ทดสอบระบบภายใต้สถานการณ์จริง เช่น ถ่ายภาพช่องปากของผู้ป่วย และให้ระบบตรวจคัดกรอง และให้ทันตแพทย์ประเมินความถูกต้องแม่นยำของระบบ เป็นต้น
    \item ให้ทันตแพทย์ประเมินประโยชน์ที่ได้รับจากระบบ เช่น ช่วยให้ตรวจคัดกรองมะเร็งช่องปากได้อย่างมีประสิทธิภาพหรือไม่ เป็นต้น

\end{itemize}


\subsection{ทันตบุคลากร}

ทันตบุคลากรมีความต้องการใช้งานระบบที่อำนวยความสะดวกในการทำงาน เช่น การดูประวัติการตรวจคัดกรองมะเร็งช่องปาก การบันทึกข้อมูล การสรุปผลการตรวจคัดกรองมะเร็งช่องปาก ดังนั้น ในการทดสอบกับทันตบุคลากร ควรเน้นการประเมินปัจจัยต่างๆ เช่น

\begin{itemize}
\item ความสะดวกในการใช้งาน: Ease of use
\item ประโยชน์: Benefits


ตัวอย่างวิธีการทดสอบกับทันตบุคลากร ได้แก่

\item ให้ทันตบุคลากรทดสอบระบบและรวบรวมข้อมูลเกี่ยวกับประสบการณ์การใช้งาน เช่น ระยะเวลาในการดำเนินการแต่ละขั้นตอน ความสะดวกในการใช้งาน เป็นต้น
\item ให้ทันตบุคลากรประเมินประโยชน์ที่ได้รับจากระบบ เช่น ระบบช่วยให้ทำงานได้อย่างมีประสิทธิภาพหรือไม่ เป็นต้น
\end{itemize}

\subsection{อาสาสมัครสาธารณสุขประจําหมู่บ้าน (อสม.)}

อสม. มีความต้องการใช้งานระบบที่เข้าใจง่าย ใช้งานสะดวก และสามารถช่วยให้ให้บริการประชาชนได้อย่างมีประสิทธิภาพ ดังนั้น ในการทดสอบกับอสม. ควรเน้นการประเมินปัจจัยต่าง ๆ เช่น
\begin{itemize}
\item ความน่าใช้งาน: Ease of use
\item ความพึงพอใจของผู้ใช้งาน: User satisfaction
\item ประโยชน์: Benefits



ตัวอย่างวิธีการทดสอบกับอสม. ได้แก่

\item ให้อสม.ทดสอบระบบและรวบรวมข้อมูลเกี่ยวกับประสบการณ์การใช้งาน เช่น ระยะเวลาในการดำเนินการแต่ละขั้นตอน ความสะดวกในการใช้งาน เป็นต้น
\item ให้อสม.ตอบแบบสอบถามเกี่ยวกับความพึงพอใจต่อระบบ เช่น ความง่ายในการใช้งาน ความน่าสนใจของเนื้อหา เป็นต้น
\item ให้อสม.ประเมินประโยชน์ที่ได้รับจากระบบ เช่น ระบบช่วยให้ให้บริการประชาชนได้อย่างมีประสิทธิภาพหรือไม่ เป็นต้น
\end{itemize}
ทั้งนี้ ในการทดสอบระบบกับผู้ใช้ทั่วไป, ทันตแพทย์, ทันตบุคลากรและอสม จะพิจารณาจากปัจจัยต่าง ๆ เช่น วัตถุประสงค์ของการประเมิน ขอบเขตของการประเมิน ความพร้อมของระบบ เป็นต้น เพื่อให้ได้ผลการประเมินที่มีประสิทธิภาพ


\ifproject
\chapter{\ifenglish Conclusions and Discussions\else บทสรุปและข้อเสนอแนะ\fi}

\section{\ifenglish Conclusions\else สรุปผล\fi}
การทําโครงงานนี้สามารถพัฒนาเว็บแอปพลิเคชั่นที่สามารถใช้งานตามความต้องการของผู้ใช้งานได้
โดยผู้ใช้งานทั้ง 3 กลุ่ม คือ
\begin{itemize}
    \item ผู้ใช้ที่เป็นผู้สร้างงานกิจกรรม (Event Manager)
          สามารถสร้างงานกิจกรรม และจัดการงานกิจกรรมได้
    \item ผู้ใช้ที่เป็นผู้นำเสนอโครงการ (Presenter)
          สามารถสร้างโครงการ และจัดการโครงการได้
    \item ผู้ใช้ที่เป็นแขกผู้เข้าร่วมงาน (Guest)
          สามารถดูงานกิจกรรมและโครงการต่าง ๆ และให้ Virtual Money และแสดงความคิดเห็นในโครงการต่าง ๆ ได้
\end{itemize}
\section{\ifenglish Challenges\else ปัญหาที่พบและแนวทางการแก้ไข\fi}
ในการทำโครงงานนี้ พบว่าเกิดปัญหาหลัก ๆ ดังนี้

\begin{itemize}
    \item การทำงานร่วมกันของทีม
          ทีมมีคนทำงานร่วมกัน 3 คน แต่เนื่องจากการประชุมและการทำงานร่วมกันไม่สม่ำเสมอ ทำให้การทำงานมีความล่าช้าอยู่บ้าง
    \item การทดสอบระบบ
          การทดสอบระบบระหว่างการพัฒนาไม่ครอบคลุม ทำให้มีบางส่วนของระบบที่ไม่ทำงานอย่างที่ควรจะเป็น
    \item การแสดงผลข้อมูล
          การแสดงผลข้อมูลตามกลุ่มผู้ใช้งานไม่ครอบคลุม ทำให้ผู้ใช้งานบางกลุ่มไม่สามารถใช้งานระบบได้อย่างที่ควรจะเป็น
\end{itemize}

\section{\ifenglish%
      Suggestions and further improvements
  \else%
      ข้อเสนอแนะและแนวทางการพัฒนาต่อ
  \fi
 }

ข้อเสนอแนะเพื่อพัฒนาโครงงานนี้ต่อไป มีดังนี้


ข้อเสนอที่ได้รับจากแบบสำรวจความพึงพอใจของผู้ใช้งานและผู้พัฒนาระบบ ได้แก่
\begin{itemize}
    \item พัฒนาส่วนแดชบฮร์ดสำหรับผู้ใช้งานที่เป็นผู้สร้างงานกิจกรรม (Event Manager) ให้มีการแสดงผลข้อมูลที่ครอบคลุมมากขึ้นในแต่ละกิจกรรม ซึ่งอาจประกอบไปด้วย จำนวนผู้เข้าร่วมงาน จำนวนโครงการที่เข้าร่วม จำนวน Virtual Money ที่แต่ละโครงการได้รับ แขกผู้เข้าร่วมงานให้ Virtual Money และแสดงความคิดเห็นในโครงการต่าง ๆ
    \item พัฒนาการแจ้งเตือนให้ผู้ใช้ที่เป็นผู้นำเสนอโครงการ (Presenter) ให้มีการแจ้งเตือนเมื่อโครงการของตนได้รับ Virtual Money หรือมีความคิดเห็นใหม่ ๆ
    \item ในส่วนของการเพิ่มโครงการเข้ามาในงานกิจกรรม ให้มีการตรวจสอบและอนุมัติ (Approve) โครงการจากผู้สร้างงานกิจกรรมก่อนที่จะแสดงผลในงานกิจกรรม
\end{itemize}

\fi

\bibliography{sampleReport}

\ifproject
\normalspacing
\appendix
% \chapter{The first appendix}

% Text for the first appendix goes here.

% \section{Appendix section}

% Text for a section in the first appendix goes here.

% test ทดสอบฟอนต์ serif ภาษาไทย

% \textsf{test ทดสอบฟอนต์ sans serif ภาษาไทย}

% \verb+test ทดสอบฟอนต์ teletype ภาษาไทย+

% \texttt{test ทดสอบฟอนต์ teletype ภาษาไทย}

% \textbf{ตัวหนา serif ภาษาไทย \textsf{sans serif ภาษาไทย} \texttt{teletype ภาษาไทย}}

% \textit{ตัวเอียง serif ภาษาไทย \textsf{sans serif ภาษาไทย} \texttt{teletype ภาษาไทย}}

% \textbf{\textit{ตัวหนาเอียง serif ภาษาไทย \textsf{sans serif ภาษาไทย} \texttt{teletype ภาษาไทย}}}

% \url{https://www.example.com/test_ทดสอบ_url}

\chapter{\ifenglish Manual\else คู่มือการติดตั้ง (สำหรับการพัฒนา)\fi}
การพัฒนาระบบของแบ่งออกเป็น 2 ส่วน คือ ฝั่งหน้าบ้าน (Front-End) และฝั่งหลังบ้าน (Back-End) จึงต้องทำการติดตั้งแยกกัน ดังนี้

\section{\ifenglish Front-End\else ฝั่งหน้าบ้าน (Front-End)\fi}
\begin{itemize}
    \item สิ่งที่เครื่องสำหรับติดตั้งต้องมีคือ Node.js และ npm โดยสามารถดาวน์โหลดได้จาก \url{https://nodejs.org/en/download/}
    \item Download source code หรือ Clone ได้จาก GitHub \url{https://github.com/Panthonf/frontend-gallery-walk.git}
    \item ทำการเปิด Terminal หรือ Command Prompt และเข้าไปที่โฟลเดอร์ที่เก็บ source code และทำการติดตั้งโดยใช้คำสั่ง \texttt{npm install} เพื่อทำการติดตั้ง package ที่จำเป็น
    \item เพิ่ม .env ไฟล์เพื่อกำหนดค่าต่าง ๆ ได้แก่
          \begin{itemize}
              \item \texttt{VITE\_BACKEND\_ENDPOINT} คือ http://localhost:8080/api/login/google ที่ใช้ในการเชื่อมต่อกับฝั่งหลังบ้าน
              \item \texttt{VITE\_CHECK\_LOGIN} คือ http://localhost:8080/api/isLoggedIn ที่ใช้ในการเชื่อมต่อกับฝั่งหลังบ้าน
              \item \texttt{VITE\_BASE\_ENDPOINTMENT} คือ http://localhost:8080/api ที่ใช้ในการเชื่อมต่อกับฝั่งหลังบ้าน
              \item \texttt{VITE\_FRONTEND\_ENDPOINT} คือ http://localhost:3000 ที่ใช้ในการเชื่อมต่อกับฝั่งหน้าบ้าน
          \end{itemize}
    \item หลังจากติดตั้งเสร็จสิ้น สามารถทำการรันโปรแกรมได้โดยใช้คำสั่ง \texttt{npm run dev} และเข้าไปที่ \url{http://localhost:3000} ในเว็บเบราว์เซอร์ จะพบกับหน้าเว็บที่ใช้ในการใช้งาน
\end{itemize}

\section{\ifenglish Back-End\else ฝั่งหลังบ้าน (Back-End)\fi}
\begin{itemize}
    \item สิ่งที่เครื่องสำหรับติดตั้งต้องมีคือ Node.js และ npm โดยสามารถดาวน์โหลดได้จาก \url{https://nodejs.org/en/download/}
    \item Download source code หรือ Clone ได้จาก GitHub \url{https://github.com/Panthonf/backend-gallery-walk.git}
    \item ทำการเปิด Terminal หรือ Command Prompt และเข้าไปที่โฟลเดอร์ที่เก็บ source code และทำการติดตั้งโดยใช้คำสั่ง \texttt{npm install} เพื่อทำการติดตั้ง package ที่จำเป็น
    \item เพิ่ม .env ไฟล์เพื่อกำหนดค่าต่าง ๆ ได้แก่
          \begin{itemize}
              \item \texttt{PORT} คือ 8080 ที่ใช้ในการเชื่อมต่อกับฝั่งหน้าบ้าน
              \item \texttt{CALLBACK\_URI} คือ http://localhost:8080/api/login/google/callback ที่ใช้ในการเชื่อมต่อกับฝั่งหน้าบ้าน
              \item \texttt{DATABASE\_URL} คือ Connection String ของ PostgreSQL
              \item \texttt{SECRET\_KEY} คือ คีย์สำหรับการเข้ารหัส Token
              \item \texttt{NODE\_ENV} คือ development สำหรับการพัฒนา และ production สำหรับการใช้งานจริง
              \item \texttt{FRONTEND\_URL} คือ http://localhost:3000
              \item \texttt{MINIO\_ENDPOINT} คือ Minio Server
              \item \texttt{MINIO\_URL} คือ Minio Server
              \item \texttt{MINIO\_ACCESSKEY} คือ Minio Access Key
              \item \texttt{MINIO\_SECRETKEY} คือ Minio Secret Key
              \item \texttt{MINIO\_PORT} คือ 443
              \item \texttt{MINIO\_USESSL} คือ true
              \item \texttt{CALLBACK\_URI\_GUEST} คือ http://localhost:8080/api/guests/login/google/callback
          \end{itemize}
    \item ทำการสร้าง Database โดยใช้คำสั่ง \texttt{npx prisma migrate dev --name init} จะทำการสร้าง Database และ Table ตามที่กำหนดไว้ในไฟล์ \texttt{schema.prisma}
    \item หลังจากติดตั้งเสร็จสิ้น สามารถทำการรันโปรแกรมได้โดยใช้คำสั่ง \texttt{npm run dev} และสามารถเรียกใช้งาน API ผ่าน Postman หรือเว็บเบราว์เซอร์ได้
\end{itemize}


\chapter{\ifenglish User Manual\else คู่มือการใช้งาน\fi}
สามารถดูวิธีการใช้งานได้จากวิดีโอที่อัพโหลดไว้ที่ \url{https://www.youtube.com/watch?v=} 



%% Display glossary (optional) -- need glossary option.
\ifglossary\glossarypage\fi

%% Display index (optional) -- need idx option.
\ifindex\indexpage\fi

\begin{biosketch}
\begin{center}
  \includegraphics[width=1.5in]{img/nuttawan.jpg}
\end{center}
\textbf{นางสาวณัฐวรรณ เรียบเรียง} เกิดเมื่อวันที่ 7 เมษายน 2545 ณ จังหวัดชลบุรี สําเร็จการศึกษาจากโรงเรียนสาธิต "พิบูลบำเพ็ญ" มหาวิทยาลัยบูรพา เข้าศึกษาที่ภาควิชาวิศวกรรมคอมพิวเตอร์ มหาวิทยาลัยเชียงใหม่ เมื่อ กรกฏาคม 2563 โดยมีความสนใจเป็นพิเศษในด้านการพัฒนาแพลตฟอร์มออนไลน์ เช่น เว็บแอพพลิเคชัน การออกแบบ UX/UI

\begin{center}
  \includegraphics[width=1.5in]{img/panthon.jpg}
\end{center}
\textbf{นายปัณฑ์ธร กันทรัพย์} เกิดเมื่อวันที่ 24 สิงหาคม 2544 ณ จังหวัดนครสวรรค์ สําเร็จการศึกษาจากโรงเรียนนวมินทราชูทิศ มัชฌิม จ.นครสวรรค์ เข้าศึกษาที่ภาควิชาวิศวกรรมคอมพิวเตอร์ มหาวิทยาลัยเชียงใหม่ เมื่อ กรกฏาคม 2563 โดยมีความสนใจเป็นพิเศษในด้านการพัฒนาแพลตฟอร์มออนไลน์ เช่น เว็บแอพพลิเคชัน

\begin{center}
  \includegraphics[width=1.5in]{img/yanatib.jpg}
\end{center}
\textbf{นายญาณาธิป ภู่สว่าง} เกิดเมื่อวันที่ 1 มีนาคม 2545 ณ จังหวัดราชบุรี สำเร็จการศึกษาจากโรงเรียนสวนกุหลาบวิทยาลัย จ.กรุงเทพมหานคร เข้าศึกษาที่ภาควิชาวิศวกรรมคอมพิวเตอร์ มหาวิทยาลัยเชียงใหม่ เมื่อ กรกฏาคม 2563 โดยมีความสนใจเป็นพิเศษในด้านการพัฒนาแพลตฟอร์มออนไลน์ เช่น เว็บแอพพลิเคชัน
\end{biosketch}
\fi % \ifproject
\end{document}
